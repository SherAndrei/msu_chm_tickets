\begin{ticket}
  \begin{task}[Интерполяция полиномом]
    $f(x)\approx \sum_{i=1}^nc_i\phi_i(x)$.
    Дано: $a\leq x_1<\ldots<x_n\leq b$. Найти $P_{n-1}(x)=a_0+a_1x+\ldots+a_{n-1}x^{n-1}$.
  \end{task}
  \begin{theorem}
    $\exists !P_{n-1}(x): P_{n-1}=f(x_i)\ \forall f(x_i),\ a\leq x_1<\ldots<x_n\leq b, \forall i=1,\ldots,n$
  \end{theorem}
  \begin{proof}
    $P_{n-1}(x)=a_0+a_1x+\ldots+a_{n-1}x^{n-1}$
    $$
      \left(\begin{array}{ccccc}
          1      & x_1    & x_1^2  & \ldots & x_1^{n-1} \\
          1      & x_2    & x_2^2  & \ldots & x_2^{n-1} \\
          \ldots & \ldots & \ldots & \ldots & \ldots    \\
          1      & x_n    & x_n^2  & \ldots & x_n^{n-1} \\
        \end{array}\right)
      \left(\begin{array}{c}
          a_1    \\
          a_2    \\
          \ldots \\
          a_n    \\
        \end{array}\right)=
      \left(\begin{array}{c}
          f(x_1) \\
          f(x_2) \\
          \ldots \\
          f(x_n) \\
        \end{array}\right)
    $$
    $$ \det A = \prod_{i\neq j}(x_i-x_j)\neq0\Rightarrow\exists!\ \hat{a}\ \forall f$$
  \end{proof}
  \begin{remark*}
    Обсуловленность матрицы $A$ довольно плохая, так вектора матрицы "слабо" линейно независимы.
    То есть решать поставленную задачу таким методом довольно неточно. Будем решать задачу иначе:
    представим базис, в котором поставленные вектора будут ортогональны.
  \end{remark*}
  Рассмотрим полиномы $\Phi_i(x)=\prod_{j\neq i}\frac{x-x_j}{x_i-x_j}$ степени $n-1$.
  Они линейно-независимы, так как
  $\begin{cases}
      \Phi_i(x_j)=0,\ i\neq j \\
      \Phi_i(x_i)=1
    \end{cases}$
  \begin{lemma*}
    $$P_{n-1}(x)=\sum_{i=1}^nf(x_i)\Phi_i(x)\defeq L_n(x)$$
  \end{lemma*}
  \begin{proof}
    Рассмотрим разность $L_n-P_{n-1}\equiv Q_{n-1}$.
    Обратим внимание, $\forall x_i,\ i=1,\ldots,n$:
    $$Q_{n-1}(x_i)=L_n(x_i)-P_{n-1}(x_i)=f(x_i)-f(x_i)=0$$
    Но $\deg Q_{n-1}\leq n-1 \Rightarrow Q_{n-1}\equiv0\Rightarrow L_n\equiv P_{n-1}$
  \end{proof}
  \begin{theorem}
    Пусть $f(x)\in C^n[a,b]$, тогда $\forall x \in [a,b]\ \exists\ \xi=\xi(x)\in[a,b]:$
    $$f(x)-L_n(x)=\frac{f^{(n)}(\xi(x))}{n!}\omega_n(x),\text{ где } w_n=\prod_i(x-x_i)$$
  \end{theorem}
  \begin{proof}
    \begin{enumerate}
      \item Для $\overline{x}=x_i,\ i=1,\ldots,n$ верно:
            $$f(x_i)-L_n(x_i)=0=\frac{f^{(n)}(\xi(\overline{x}))}{n!}\underbrace{\prod_i(\overline{x}-x_i)}_{0}$$
      \item Для зафиксированного $\overline{x}\neq x_i,\ i=1,\ldots,n,\ \overline{x}\in[a,b]$ рассмотрим
            \begin{multline*}
              \phi(t)=f(t)-L_n(t)-k\omega_n(t),\\ \omega_n(t)=\prod_{i=1}^n(t-x_i),\ k=\frac{f(\overline{x})-L_n(\overline{x})}{\prod_i(\overline{x}-x_i)}=\const
            \end{multline*}
            Заметим, что $\phi(x_i)=0,\ \forall i$ и $\phi(\overline{x})=0\Rightarrow\exists\ n+1$ нуль. Также $\phi(x)\in C^n[a,b]$, т.к. $f, L_n, \omega_n\in C^n[a,b]$.
            Значит $\phi'(x)$ имеет $n$ нулей, $\ldots$, $\phi^{(n)}$ имеет 1 нуль, то есть $\exists\ \xi=\xi(x):\ \phi^{(n)}(\xi)=0$:
            \begin{multline*}
              f^{(n)}(t)\Bigr|_{t=\xi}-\underbrace{L_n^{(n)}(t)\Bigr|_{t=\xi}}_{0,\ \deg L_n=n-1}-k\underbrace{\omega_n^{(n)}(t)\Bigr|_{t=\xi}}_{n!}=0\\\Rightarrow k=\frac{f^{(n)}(\xi(x))}{n!},\ x\in[a,b]
            \end{multline*}
    \end{enumerate}
  \end{proof}
  \begin{corollary}
    $$\left\Vert f-L_n\right\Vert_{C[a,b]}\leq\frac{\Vert f^{(n)}\Vert_{C[a,b]}}{n!}\Vert\omega_n\Vert_{C[a,b]} $$
  \end{corollary}
  \begin{task}[Минимизация остаточного члена погрешности]
    Рассмотрим класс функций $$\mathcal{F}=\{f:f\in C^n[a,b],\ \Vert f^{(n)}\Vert_C\leq A_n\}$$
    Для набора узлов $\{x_i\}$ определим соответственно погрешность интерполяции для функции $f$ и для класса $\mathcal{F}$:
    $$l(f, \{x_i\})=\Vert f-L_n\Vert,\ l(\mathcal{F}, \{x_i\})=\sup_{f\in\mathcal{F}}l(f,\{x_i\})$$
    Надо найти отпимальный набор узлов $\{\overline{x}_i\}$:
    $$\inf_{\{x_i\}}l(\mathcal{F}, \{x_i\})=l(\mathcal{F}, \{\overline{x}_i\})$$
  \end{task}
  Возьмем некоторый набор $\{x_i\}$. Тогда
  \begin{align*}
    l(\mathcal{F}, \{x_i\})=\sup_{f\in\mathcal{F}}\left\Vert\frac{f^{(n)}(\xi(x))}{n!}\omega_n(x)\right\Vert\leq\frac{A_n}{n!}\Vert\omega_n\Vert\Rightarrow \\
    \inf_{\{x_i\}}l(\mathcal{F}, \{x_i\})\leq\frac{A_n}{n!}\inf_{\{x_i\}}\Vert\omega_n\Vert
  \end{align*}
  Решением задачи
  $$\inf_{\{x_i\}}\Vert\omega_n\Vert=\inf_{\{x_i\}}\max_{x\in[a,b]}|(x-x_1)\ldots(x-x_n)|$$
  является нормированный многочлен Чебышева, $x_i$ - его корни: $x_i=\frac{a+b}{2}+\frac{b-a}{2}\cos\frac{\pi(2i-1)}{2n}$. При этом
  $$w_n(x)=\frac{(b-a)^n}{2^{2n-1}}T_n\left(\frac{2x-(a+b)}{b-a}\right),\ \Vert \omega_n\Vert=\frac{(b-a)^n}{2^{2n-1}}$$
  Это приводит к оценке
  $$\inf_{\{x_i\}}l(\mathcal{F}, \{x_i\})\leq\frac{A_n}{n!}\frac{(b-a)^n}{2^{2n-1}}$$
  Результат оптимизации на классе не лучше, чем результат оптимизации на одном из элементов класса.
  Возмьмем $f_0\in\mathcal{F}:\ f_0=\frac{A_n}{n!}x^n$. Для него получаем оценку:
  $$\inf_{\{x_i\}}l(\mathcal{F}, \{x_i\})\geq\inf_{\{x_i\}}l(f_0, \{x_i\})=\frac{A_n}{n!}\frac{(b-a)^n}{2^{2n-1}}$$
  т.е. найденная для класса оценка сверху достигается на функции $f_0(x)$, т.е. является точной.
  Таким образом, интерполяция по узлам Чебышёва оптимальна на классе $\mathcal{F}$.
\end{ticket}
