\begin{task}
  Найти все решения задачи на собственные значения:
  \[\begin{cases}
      \frac{y_{k+1}-2y_k+y_{k-1}}{h^2} = -\lambda y_k,\ h = \frac{2}{2N-1},\ 1 \leq k \leq N-1 \\
      y_0 = 0                                                                                  \\
      y_N = -y_{N-1}
    \end{cases}\]
  \begin{enumerate}
    \item Запишем канонический вид. Найдем коэффициенты для краевых условий
          \begin{align*}
            k & = 1: \frac{y_2 - 2y_1}{h^2} = -\lambda y_1                                                             \\
            k & = N-1: \frac{y_{N} - 2y_{N-1}+y_{N-2}}{h^2} = \frac{-3\cdot y_{N-1} + y_{N-2}}{h^2} = -\lambda y_{N-1}
          \end{align*}
          Таким образом задачу можно переписать в матричном виде:
          \[\underbrace{\left(\begin{array}{cccccc}
                \frac{-2}{h^2} & \frac{1}{h^2}  &               &               &                & 0              \\
                \frac{1}{h^2}  & \frac{-2}{h^2} & \frac{1}{h^2} &               &                &                \\
                               &                & \cdots        & \cdots        &                &                \\
                               &                &               & \frac{1}{h^2} & \frac{-2}{h^2} & \frac{1}{h^2}  \\
                0              &                &               &               & \frac{1}{h^2}  & \frac{-3}{h^2} \\
              \end{array}\right)}_{N-1\times N-1}
            \left(\begin{array}{c}
                y_{1}   \\
                \\
                \vdots  \\
                \\
                y_{N-1} \\
              \end{array}\right)
            =
            -\lambda
            \left(\begin{array}{c}
                y_{1}   \\
                \\
                \vdots  \\
                \\
                y_{N-1} \\
              \end{array}\right)
          \]

    \item Для более удобного решения сделаем замену $p=1-h^2\frac{\lambda}{2}$ и перепишем условие:
          \[\begin{cases}
              y_{k+1}-2py_k+y_{k-1} = 0,\ 1 \leq k \leq N-1 \\
              y_0 = 0                                       \\
              y_N = -y_{N-1}
            \end{cases}\]
          Решим получененную разностную задачу:
          \[P(\mu) = \mu^2-2p\mu+1 = 0 \Leftrightarrow \mu_{1,2}=p \pm \sqrt{p^2-1}\]
          Также по теореме Виета:
          \begin{align}
            \mu_1\cdot \mu_2      & =1  \\
            \frac{\mu_1+\mu_2}{2} & = p
          \end{align}
          \begin{enumerate}
            \item $\mu_1\neq\mu_2$. Тогда общее решение имеет вид
                  \[y_k=C_1\mu_1^k+C_2\mu_2^k\]
                  Подставим в начальные условия, чтобы найти $C_1$ и $C_2$:
                  \[\begin{cases}
                      y_0 = 0 = C_1+C_2\Rightarrow C_2=-C_1 \\
                      y_N=-y_{N-1}: C_1\mu_1^N+C_2\mu_2^N = -(C_1\mu_1^{N-1}+C_2\mu_2^{N-1})
                    \end{cases}\]
                  Преобразуем второе равенство, используя первое:
                  \[C_1(\mu_1^N-\mu_2^N)=-C_1(\mu_1^{N-1}-\mu_2^{N-1})\]
                  Если $C_1 = 0$, то $C_2=0$, то $y_k\equiv 0$, что нам неинтересно, так как нулевой вектор не является собственным. Иначе
                  \[\mu_1^N-\mu_2^N=\mu_2^{N-1}-\mu_1^{N-1} \Leftrightarrow
                    \mu_1^N+\mu_1^{N-1}=\mu_2^N+\mu_2^{N-1} \Leftrightarrow
                    \mu_1^{N-1}(\mu_1+1)=\mu_2^{N-1}(\mu_2+1)\]
                  Используем п.1 из теоремы Виета:
                  \[\mu_1^{N-1}(\mu_1+\mu_1\mu_2)=\mu_2^{N-1}(\mu_2+1)\Leftrightarrow\mu_1^N(\mu_2+1)=\mu_2^{N-1}(\mu_2+1)\]
                  Заметим, что $\mu_2\neq 1$, так как по теореме Виета $\mu_1=\mu_2=1$ -- противоречие с предположением $\mu_1\neq\mu_2$.
                  \[\frac{\mu_1^N}{\mu_2^{N-1}}=1\Leftrightarrow\mu_1^{2N-1}=1\]
                  Возьмем $2N-1$ комплексный корень из 1 и получим:
                  \[\begin{cases}
                      \mu_1^{(m)} = \exp\left(\frac{2m\pi i}{2N-1}\right)  \\
                      \mu_2^{(m)} = \exp\left(-\frac{2m\pi i}{2N-1}\right) \\
                    \end{cases} m = 0, ...,2N-2\Leftrightarrow
                    \begin{cases}
                      \mu_1^{(m)} = \exp\left(\frac{2(m-1)\pi i}{2N-1}\right)  \\
                      \mu_2^{(m)} = \exp\left(-\frac{2(m-1)\pi i}{2N-1}\right) \\
                    \end{cases} m = 1, ...,2N-1\]
                  Решение имеет вид:
                  \begin{multline*}
                    y_k=C_1\mu_1^k+C_2\mu_2^k=2C\left(\frac{\exp\left(\frac{2(m-1)\pi i}{2N-1}\right)-\exp\left(-\frac{2(m-1)\pi i}{2N-1}\right)}{2}\right) = C\sin\frac{2(m-1)\pi k}{2N-1}\\
                    m=1,...,2N-1,\ k=1,...,N-1
                  \end{multline*}
                  Найдем собственные значения. По теореме Виета:
                  \begin{multline*}
                    p=\frac{\mu_1+\mu_2}{2}=\frac{\exp\left(\frac{2(m-1)\pi i}{2N-1}\right) + \exp\left(-\frac{2(m-1)\pi i}{2N-1}\right)}{2}=\cos\left(\frac{2(m-1)\pi}{2N-1}\right)\\
                    m=1,...,2N-1
                  \end{multline*}
                  \begin{multline*}
                    p=1-\lambda\frac{h^2}{2}\Leftrightarrow\\\lambda=\frac{2}{h^2}\left(1-\cos\left(\frac{2(m-1)\pi}{2N-1}\right)\right)=\frac{4}{h^2}\sin^2\left(\frac{(m-1)\pi}{2N-1}\right)\\
                    m=1,...,2N-1
                  \end{multline*}
                  У матрицы размера $N-1\times N-1$ не может быть больше $N-1$ собственного значения, но выше мы получили $2N-1$. Отберем корни.
                  Заметим, что при $m=1$ $\lambda=0$, такое собственное значение порождает нулевой вектор, так как матрица невырождена, а нулевых
                  собственных векторов нет по определению собственного вектора. Осталось $2N-2$. Покажем,
                  что корни симметричны.

                  Обозначим $\alpha_m=\frac{\pi(m-1)}{2N-1}$. Тогда
                  \begin{align*}
                    \alpha_2      & = \frac{\pi}{2N-1}                                \\
                    \alpha_3      & = \frac{2\pi}{2N-1} = 2\alpha_1                   \\
                    ...           &                                                   \\
                    \alpha_{2N-2} & = \frac{\pi(2N-3)}{2N-1}= \pi - \frac{2\pi}{2N-1} \\
                    \alpha_{2N-1} & = \frac{\pi(2N-2)}{2N-1}= \pi - \frac{\pi}{2N-1}  \\
                  \end{align*}
                  То есть углы симметричны относительно $\frac{\pi}{2}$, а значит количество различных корней ровно $N-1$.

            \item $\mu_1 = \mu_2$: из теоремы Виета следует
                  $\mu_1=\mu_2=p$ и $\mu_1\mu_2 = 1$ $\Rightarrow$ $\lambda=0$ и $\lambda=\frac{4}{h^2}$.
                  Вставим это решение в ответ из случая $\mu_1\neq\mu_2$
          \end{enumerate}
          Итоговый ответ:
          \begin{align*}
            y_k     & = C\sin\frac{2m\pi k}{2N-1}                         & m=1,...,N-1 \\
            \lambda & = \frac{4}{h^2}\sin^2\left(\frac{m\pi}{2N-1}\right) & k=0,...,N
          \end{align*}
  \end{enumerate}
\end{task}
