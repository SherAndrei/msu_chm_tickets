\begin{task}
  Найдем ограниченное фундаментальное решение задачи
  \[y_{k-1}-2y_k+y_{k+1}=\sigma_k^0\]
  Так как задача неоднородная, то решение можно представить в виде
  $y_k=y_k^o+y_k^1$.
  Найдем общее решение задачи. Для этого решим $y_{k-1}-2y_k+y_{k+1}=0$
  \[P(\mu)=1-2\mu+\mu^2=0\Rightarrow\mu=1\text{ кратности 2}\Rightarrow y_k=C_1+C_2k\]
  Частное решение будем строить с помощью хитрого трюка. Запишем
  нашу задачу в виде системы
  \[\begin{cases}
      y_{k-1}-2y_k+y_{k+1}=0,\ k\leq -1                                  \\
      y_{k-1}-2y_k+y_{k+1}=1,\ k = 0 \Leftrightarrow y_{-1}-2y_0+y_{1}=1 \\
      y_{k-1}-2y_k+y_{k+1}=0,\ k \geq 1                                  \\
    \end{cases}\]

  Так как мы ищем частное решение задачи (ищем любое),
  то ради удобства занулим решение $\forall\ k\leq-1$.

  Ищем $y_0$: подставим в первое уравнение $k=-1$, тогда
  \[y_{-2}-2y_{-1}+y_{0}=0\Rightarrow y_0=0\]

  Теперь мы можем найти $y_1$, посмотрев на второе уравнение:
  \[y_{-1}-2y_0+y_{1}=1\Rightarrow y_1=1\]

  Найдем $y_2$, чтобы восстановить константы для правой части, подставив $k=1$ в третье уравнение:
  \[y_{0}-2y_1+y_{2}=0\Rightarrow y_2=2\]

  Для того чтобы найти правую часть частного решения подставим
  в известное нам общее решение:
  \[\begin{cases}
      1=C_1^++C_2^+ \\
      2=C_1^++C_2^+\cdot2
    \end{cases}\Rightarrow\begin{cases}
      C_1^+=0 \\
      C_2^+=1
    \end{cases}\Rightarrow y_k=k,\ k\geq1\]

  Таким образом искомое решение принимает вид
  \[y_k=C_1+C_2k + \begin{cases}
      0,\ k\leq 0 \\
      k,\ k\geq1
    \end{cases}= \begin{cases}
      C_1+C_2k,\ k\leq 0 \\
      C_1+(C_2+1)k,\ k\geq1
    \end{cases}\]
  $\forall\ C_1,\ C_2$ решение не будет ограниченным, ограниченных решений нет.
\end{task}
