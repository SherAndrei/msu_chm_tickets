\begin{task}
  Среди всех многочленов вида $5x^3+a_2x^2+a_1x+a_0$
  найти наимее уклоняющийся от нуля на отрезке $[1,2]$.

  \begin{enumerate}
    \item В классе многочленов степени $n$ удовлетворяющих
          условию $P_n^{(k)}(0)=c\cdot k!\neq0$
          наименее уклоноящийся от 0 на $[a,b]$ имеет вид:
          \[P_n^*(x)=ck!\left(\frac{b-a}{2}\right)^k\frac{T_n\left(\frac{2x-(a+b)}{b-a}\right)}{T_n^{(k)}\left(\frac{a+b}{a-b}\right)}\]
          В условиях нашей задачи $c=5,\ k=3,\ n=3,\ a=1,\ b=2$
          \[T_0=1,\ T_1=x,\ T_2=2x^2-1,\ T_3=2x(2x^2-1)-x=4x^3-3x,\ T_3^{(3)}=4\cdot3!\]
          \[P_n^*(x)=5\cdot 3!\left(\frac{1}{2}\right)^3\frac{T_3\left(2x-3\right)}{4\cdot3!}=\frac{5}{32} T_3(2x-3)\]
    \item Докажем, что такой многочлен действительно наименее уклоняющийся.

          Пусть $\exists \tilde{P}_n^*:\ \norm{\tilde{P}_n^*}<\norm{P_n^*}$. Рассмотрим $Q_n=\tilde{P}_n^*-P_n^*$
          \[\sgn Q_n|_{x_{(m)}}=(-1)^m,\ m=0,\ldots,n\]
          То есть $Q_n$ имеет $n+1$ экстремум, то есть $n$ различных корней.
          Значит $Q_n^{(k)}$ имеет $n-k$ корней на $[a,b]$, но помимо них есть еще корень $0$,
          так как коэффициенты при $x^k$ совпадают, так как $\tilde{P}_n^*,\ P_n^*$ из одного класса.
          Таким образом $Q_n^{(k)}\equiv 0\Rightarrow Q_n$ - многочлен $k-1$ степени, но
          корней у него $n$, значит $Q_n\equiv0$.
  \end{enumerate}

  Ответ: $\frac{5}{32} T_3(2x-3)=\frac{5}{32}(32x^3-144x^2+210x-99)$
\end{task}
