\begin{task}
  Для задачи
  \[-y''(x) +2y(x) = f(x),\ y(0) = 1,\ y'(1) = 0 \]
  Построить разностную схему второго порядка аппроксимации на
  сетке $x_i=ih,\ i=0,\ldots,N,\ h=\frac{1}{N}$. Исследовать устойчивость.

  В качестве разностного уравнения для разностной схемы предлагается брать
  \[-\frac{y_{k+1}-2y_k+y_{k-1}}{h^2}+2y_k = f_k,\ h = \frac{1}{N},\ 1 \leq k \leq N-1\]
  Такое уравнение обеспечит второй порядок аппроксимации, что мы проверим позже.
  Осталось подобрать краевые условия для схемы: для $y(0)=1$ обеспечит аппроксимацию
  любого порядка точное значение $y_0=1$, тогда как для $y'(1)=0$ нужно
  подобрать что-то особенное: знаем, что $\frac{y_k-y_{k-1}}{h}$ дает
  первый порядок аппроксимации, предлагается использовать $\delta$-поправку,
  чтобы получить второй. Будем искать дополнительное слагаемое из условия аппроксимации
  \begin{multline*}
    \abs{\frac{y(1)+y(1-h)}{h}-\delta}=\abs{\frac{y(1)+y(1)-hy'(1)+\frac{h^2}{2}y''(1)+\bigO(h^3)}{h}-\delta}=\abs{\frac{h}{2}y''(1)+\bigO(h^2)-\delta}\leq ch^2\Leftrightarrow\delta=\frac{h}{2}y''(1)= \\
    \frac{h}{2}(2y(1)-f(1))
  \end{multline*}
  Тогда краевое условие будет иметь вид
  \[\frac{y_N-y_{N-1}}{h}+\frac{h}{2}(2y_N-f_N)=0\Leftrightarrow 2\frac{y_N-y_{N-1}}{h^2}+2y_N=f_N\]
  Итоговая разностная схема имеет вид
  \[\begin{cases}
      -\frac{y_{k+1}-2y_k+y_{k-1}}{h^2}+2y_k = f_k,\ h = \frac{1}{N},\ 1 \leq k \leq N-1 \\
      x_k = kh,\ f_k := f(x_k),\ p_k := p(x_k)                                           \\
      y_0 = 1                                                                            \\
      2\frac{y_N-y_{N-1}}{h^2}+2y_N=f_N
    \end{cases}\]
  \[k=1: \frac{y_{2}-2y_1+1}{h^2}+2y_1 = f_1\]
  Задача в матричном виде имеет вид:
  \[
    -\left(\begin{array}{cccccc}
        \frac{-2}{h^2} & \frac{1}{h^2}  &               &               &                & 0              \\
        \frac{1}{h^2}  & \frac{-2}{h^2} & \frac{1}{h^2} &               &                &                \\
                       &                & \cdots        & \cdots        &                &                \\
                       &                &               & \frac{1}{h^2} & \frac{-2}{h^2} & \frac{1}{h^2}  \\
        0              &                &               &               & \frac{2}{h^2}  & \frac{-2}{h^2} \\
      \end{array}\right)
    \left(\begin{array}{c}
        y_{1}  \\
        \\
        \vdots \\
        \\
        y_{N}  \\
      \end{array}\right)
    +
    2
    \left(\begin{array}{c}
        y_{1}  \\
        \\
        \vdots \\
        \\
        y_{N}  \\
      \end{array}\right)
    =
    \left(\begin{array}{c}
        f_{1}  \\
        \\
        \vdots \\
        \\
        f_{N}  \\
      \end{array}\right)
  \]
  Будем использовать сеточную интегральную норму
  \[\Vert y_h\Vert^2_h = (y_h,y_h)_h= \sum_{k=1}^{N-1}y^2_kh\]
  согласованной с нормой $\Vert y(x)\Vert^2_{L^2(0,1)} = \int_0^1y^2(x)dx$ исходной задачи.

  План:
  \begin{enumerate}[I.]
    \item Доказать, что разностная схема имеет порядок аппроксимации $O(h^2)$.
    \item Доказать устойчивость разностный схемы.
    \item Доказать, что есть сходимость $O(h^2)$.
  \end{enumerate}

  \newpage

  \begin{enumerate}[I.]
    \item Докажем аппроксимацию второго порядка на решении:
          \begin{enumerate}
            \item $\Vert L_h(y)_{Y_h}-f_h\Vert_{F_h}\leq\bigO(h^2)$
                  \[\max_{x_k}\left|-\frac{y(x_k+h)-2y(x_k)+y(x_k-h)}{h^2}+2y(x_k)-f(x_k)\right|=\]
                  \[y(x_k\pm h)=y(x_k)\pm hy'(x_k)+\frac{h^2}{2}y''(x_k)\pm\frac{h^3}{6}y'''(x_k)+\bigO(h^4)\]
                  \[=\max_{x_k}\left|-\frac{h^2y''(x_k)+\bigO(h^4)}{h^2}+2y(x_k)-f(x_k)\right|=\]
                  \[=\max_{x_k}\left|-y''(x_k)+\bigO(h^2)+2y(x_k)-f(x_k)\right|=\bigO(h^2)\]
            \item $\Vert l_h(y)_{Y_h}-\varphi_h\Vert_{\Phi_h}\leq\bigO(h^2)$
                  \[y_0 = 1:\ \Vert y(0)-1\Vert_{\Phi_h}=0\]
                  \begin{multline*}
                    \frac{y_N - y_{N-2}}{2h} = 0:\ \norm{\frac{y(1)-y(1-2h)}{2h}}_{\Phi_h}= \\
                    y\left(1-2h\right)=y(1)-2hy'(1)+\frac{4h^2}{2}y''(x_k)+\bigO(h^3) \\
                    =\norm{-y'(1)+2hy''(x_k)+\bigO(h^2)}_{\Phi_h}=\bigO(h^2)
                  \end{multline*}
            \item Условия нормировки
                  \begin{align*}
                    \lim_{h\rightarrow0}\Vert f_h-(f)_{F_h}\Vert_{F_h}                   & =0 \Rightarrow f(x_k)-f(x_k) = 0       \\
                    \lim_{h\rightarrow0}\Vert \varphi_h-(\varphi)_{\Phi_h}\Vert_{\Phi_h} & =0 \Rightarrow (0\ 0)^T - (0\ 0)^T = 0
                  \end{align*}
          \end{enumerate}
          Значит схема имеет \textbf{второй порядок аппроксимации}.

          \textbf{Замечание}: Доказали аппроксимацию на решении в $\Vert\cdot\Vert_{\infty}$, но
          \[\Vert x\Vert_h=\sqrt{\sum_{i=1}^{N-1}x_i^2h}\leq\max_i|x_i|\sqrt{\sum_{i=1}^{N-1}h}\leq\max_i|x_i|\sqrt{\frac{2(N-1)}{2N-1}}\leq\max_i|x_i|\cdot1=\Vert x\Vert_{\infty}\]
          То есть из аппроксимации в $\Vert\cdot\Vert_{\infty}$ следует аппроксимация в $\Vert\cdot\Vert_h$.

          \newpage

    \item Напомним определение устойчивости разностной схемы.
          \begin{definition}
            Пусть уравнение $y''(x)=f(x)$ доопределено краевыми
            условиями на разных концах отрезка. Разностная схема
            $A_hy_h = f_h$ линейной задачи устойчива, если существуют $C$, $h_0$ такие, что для
            произвольных $A_hy^{(1,2)}_h = f^{(1,2)}_h$ выполняется оценка
            \[\Vert y^{(1)}_h-y^{(2)}_h\Vert_h\leq C\Vert f^{(1)}_h -f^{(2)}_h\Vert_h\ \forall h\leq h_0\]
            с константой $C$, не зависящей от $h$.
          \end{definition}

          Будем доказывать устойчивость разностной схемы энергетическим методом.
          Запишем нашу дифференциалную задачу
          \[-y''(x)+p(x)y(x)=f(x),\ y(0) = y'(1) = 0,\ p(x)\geq 0\]
          Умножим уравнение на $y(x)$, и результат проинтегрируем по отрезку $[0, 1]$
          \[\int_0^1 (-y''y+py^2)dx = \int_0^1fydx \]
          \[\int_0^1 -y''ydx+ \int_0^1py^2 dx = \int_0^1fydx \]
          Проинтегрируем по частям первое слагаемое
          \[\int_0^1 -y''ydx = \int_0^1-ydy' = -yy'\vert^1_0 - \int_0^1y'd(-y) = \int_0^1(y')^2dx\]
          Получили интегральное тождество
          \[\int_0^1 (y'(x))^2dx+ \int_0^1py^2 dx = \int_0^1fydx \]
          Оценим слева через неравенство, связывающее интегралы от квадратов
          функции и ее производной. Так как $y(0) = 0$, то справедливо следующее:
          \[y(x_0) = \int_0^{x_0}y'(x)dx\]
          Применим интегральную форму неравенства Коши-Буняковского:
          \[|y(x_0)|^2 = \left|\int_0^{x_0}y'dx\right|^2\leq\left(\int_0^{x_0}1^2dx\right)\left(\int_0^{x_0}(y')^2dx\right)\leq\int_0^{x_0}(y')^2dx\leq\int_0^{1}(y')^2dx\]
          После интегрирования по $x_0$ по отрезку $[0,1]$ обеих частей получим искомое равенство
          \[\int_0^1|y(x_0)|^2dx_0 \leq \int_0^{1}(y')^2dx\int_0^1dx_0 \Leftrightarrow \int_0^1y^2dx\leq\int_0^1(y')^2dx\]
          Оценку справа выведем из разности квадратов:
          \[0\leq\int_0^1(f - y)^2dx\leq\int_0^1f^2dx-2\int_0^1fydx+\int_0^1y^2dx\]
          \[\Rightarrow\int_0^1fydx\leq\frac{1}{2}\left(\int_0^1f^2dx + \int_0^1y^2dx\right)\]

          Таким образом имеем:
          \[\int_0^1y^2dx\leq\int_0^1 (y'(x))^2dx+ \int_0^1py^2 dx = \int_0^1fydx\leq\frac{1}{2}\left(\int_0^1f^2dx + \int_0^1y^2dx\right)\]
          Таким образом верна оценка
          \[\int_0^1y^2dx\leq\int_0^1f^2dx\Rightarrow\Vert y\Vert_{L_2(0,1)}\leq\Vert f\Vert_{L_2(0,1)}\]
          Это означает устойчивость дифференциальной задачи по правой части.

          Докажем теперь устойчивость разностной схемы.
          \[-\frac{y_{k+1}-2y_k+y_{k-1}}{h^2}+p_ky_k = f_k,\ 1 \leq k \leq N-1,\ y_0 = 0,\ y_N = y_{N-1}\]
          Умножим на $y_k$ и просуммируем от $1$ до $N-1$. Так как $y_0 = 0,\ y_N = y_{N-1}$
          \[-\frac{1}{h^2}\left(\sum_{k=1}^{N-1}\left(y_{k+1}-2y_k+y_{k-1}\right)y_k\right)=-\frac{1}{h^2}\left(\sum_{k=1}^{N-1}\left(y_{k+1}-y_k-y_k+y_{k-1}\right)y_k\right)=\]
          \[=-\frac{1}{h^2}\sum_{k=1}^{N-1}\left(y_{k+1}-y_k\right)y_k+\frac{1}{h^2}\sum_{k=1}^{N-1}\left(y_k-y_{k-1}\right)y_k=\]
          \[=-\frac{1}{h^2}\sum_{k=2}^{N}\left(y_{k}-y_{k-1}\right)y_{k-1}+\frac{1}{h^2}\sum_{k=1}^{N-1}\left(y_k-y_{k-1}\right)y_k=\frac{1}{h^2}\sum_{k=1}^{N}(y_k-y_{k-1})^2\]
          Получили конечномерный аналог интегрального тождества:
          \[\frac{1}{h^2}\sum_{k=1}^N(y_k-y_{k-1})^2+\sum_{k=1}^{N-1}p_ky_k^2=\sum_{k=1}^{N-1}f_ky_k\]
          Для оценки слева докажем сеточный аналог неравенства для функции и ее производной в точках $k=1,...,N-1$.
          Так как $y_0 = 0$, справедливо следующее:
          \[y_k=\sum_{i=1}^{k}(y_i-y_{i-1})\]
          Воспользуемся неравенством Коши-Буняковского и $y_N=y_{N-1}$
          \[y_k^2\leq\left(\sum_{i=1}^k1^2\right)\left(\sum_{i=1}^k(y_i-y_{i-1})^2\right)\leq (N-1)\sum_{i=1}^{N-1}(y_i-y_{i-1})^2\]
          Суммируя до $N-1$ обе части, при $h=\frac{2}{2N-1}$ получаем оценку:
          \[\sum_{k=1}^{N-1}y_k^2\leq(N-1)^2\sum_{k=1}^{N-1}(y_k-y_{k-1})^2\leq\frac{1}{h^2}\sum_{k=1}^{N-1}(y_k-y_{k-1})^2\]
          Найдем аналогично дифференциальному неравенству оценку справа
          \[0\leq\sum_{k=1}^{N-1}(f_k-y_k)^2 = \sum_{k=1}^{N-1}f_k^2-2\sum_{k=1}^{N-1}f_ky_k+\sum_{k=1}^{N-1}y_k^2\]
          \[\Rightarrow\sum_{k=1}^{N-1}f_ky_k\leq\frac{1}{2}\left(\sum_{k=1}^{N-1}f_k^2+\sum_{k=1}^{N-1}y_k^2\right)\]
          Итоговая оценка имеет вид
          \[\sum_{k=1}^{N-1}y_k^2\leq\frac{1}{h^2}\sum_{k=1}^{N-1}(y_k-y_{k-1})^2+\sum_{k=1}^{N-1}p_ky_k^2=\sum_{k=1}^{N-1}f_ky_k\leq\frac{1}{2}\left(\sum_{k=1}^{N-1}f_k^2+\sum_{k=1}^{N-1}y_k^2\right)\]
          Таким образом
          \[\sum_{k=1}^{N-1}y_k^2\leq\sum_{k=1}^{N-1}f_k^2\Rightarrow\sum_{k=1}^{N-1}y_k^2h\leq\sum_{k=1}^{N-1}f_k^2h\Rightarrow\Vert y_h\Vert^2_h\leq\Vert f_h\Vert^2_h\]
          То есть \textbf{разностная схема устойчива} в норме $\Vert\cdot\Vert_h$.
          \newpage
    \item Докажем, что у схемы есть сходимость порядка $\bigO(h^2)$.
          \begin{theorem}[Филиппов А.Ф.]
            Пусть выполнены следюущие условия:
            \begin{enumerate}
              \item операторы $L$, $l$ и $L^h$, $l^h$ - линейные;
              \item решение дифференциальной задачи $\exists!$;
              \item разностная схема аппроксимирует на решении дифференциальную задачу с порядком $p$;
              \item разностная схема устойчива;
            \end{enumerate}
            Тогда решение разностной схемы сходится к решению дифференциальной задачи с порядком не ниже $p$
          \end{theorem}
          Посмотрим на наши результаты
          \begin{enumerate}
            \item операторы $L$, $l$ и $L^h$, $l^h$ - действительно линейные;
            \item решение дифференциальной задачи $\exists!$, так как по условию $y$ и $f$ хорошие гладкие функции.
            \item разностная схема аппроксимирует на решении дифференциальную задачу с порядком $2$;
            \item разностная схема действительно устойчива;
          \end{enumerate}
          Таким образом решение разностной схемы \textbf{сходится} к решению дифференциальной задачи \textbf{с порядком не ниже $2$}.
  \end{enumerate}
\end{task}
