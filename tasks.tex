\documentclass[a4paper]{article}

%Russian-specific packages
%--------------------------------------
\usepackage[T2A]{fontenc}
\usepackage[utf8]{inputenc}
\usepackage[english, russian]{babel}
%for search in russian
\usepackage{cmap}
%--------------------------------------

%Math-specific packages
%--------------------------------------
\usepackage{amsmath}
\usepackage{amssymb}

%Format-specific packages
%--------------------------------------
\usepackage[left=1cm,
            right=1cm,
            top=1cm,
            bottom=1cm,
            bindingoffset=0cm]{geometry}
%--------------------------------------

% for theorems, lemmas and definitions
%--------------------------------------
\usepackage{amsthm}

\newtheorem{theorem}{Теорема}[section]

\theoremstyle{definition}
\newtheorem {definition}{Опр.}[section]
\newtheorem {task}{Задача}[section]

%--------------------------------------

%Roman enum items
\usepackage{enumerate}

% For graphics
%--------------------------------------
\usepackage{tikz}
\usetikzlibrary{
  % for faster compilation
  external
  % for cool arrows
  , arrows.meta
  % for angles
  , angles
  , quotes
  , babel
}
\tikzexternalize

%--------------------------------------

% My commands
%--------------------------------------

\DeclareMathOperator{\sgn}{sgn}

\def\const{ \mathrm{const} }
\def\eps{ \varepsilon }
\def\Eps{ \mathcal{E} }

\def\R{ \mathbb{R} }
\def\Z{ \mathbb{Z} }
\def\C{ \mathbb{C} }
\def\E{ \mathrm{E} }
\def\D{ \mathrm{D} }
\def\P{ \mathrm{P} }

\def\littleO{ \overline{\overline{o}} }
\def\bigO{ \underline{\underline{\mathcal{O}}} }

\newcommand*{\norm}[1]{\left\lVert#1\right\rVert}
\newcommand*{\abs}[1]{\left\lvert#1\right\rvert}

% suppress page count
\pagestyle{empty}

\includeonly{
  task01/task,
  task02/task,
  task03/task,
  task04/task,
  task05/task,
  task06/task,
  task07/task,
  task08/task,
  task09/task,
  task10/task,
  task11/task,
  task12/task,
  task13/task,
  task14/task,
  task15/task,
}

\begin{document}

\begin{task}
  Найти общее действительное решение уравнения
  \[13y_{k-1}+4y_k+y_{k+1}=0\]
  Будем искать решение в виде $y_k=\mu^k$, расссмотрим характеристический
  многочлен
  \[P(\mu)=13\mu^{k-1}+4\mu^k+\mu^{k+1}=0\Leftrightarrow13+4\mu+\mu^2=0\]
  \[D=16-4*13=-36=-1\cdot6^2\Rightarrow\mu_{1,2}=\frac{-4\pm i\cdot6}{2}=-2\pm 3i\]
  Запишем решение в действительной форме
  \[\Re(\mu_{1,2})=-2,\ \Im(\mu_{1,2})=\pm3\]

  \begin{minipage}{.3\linewidth}
    \tikzsetnextfilename{task01/Roots}
    \begin{tikzpicture}[scale=0.6]
      \coordinate (O) at (0,0);
      \draw[step=1cm,gray,very thin] (-3,-3) grid (1,3);
      \draw[thick,->] (0,-3) -- (0,3) node[left]{$\Im$};
      \draw[thick,->] (-3,0) -- (1,0) coordinate (Re) node[right]{$\Re$};
      \draw[thick,-Circle] (O) -- (-2,3) coordinate (mu1) node[below]{$\mu_1$};
      \draw[thick,-Circle] (O) -- (-2,-3) node[above]{$\mu_2$};
      \pic [draw, angle radius=0.5cm,"$\varphi$"] {angle=Re--O--mu1};
    \end{tikzpicture}
  \end{minipage}\hfill
  \begin{minipage}{.7\linewidth}
    \[\rho=\sqrt{\Re(\mu_{1,2})^2 + \Im(\mu_{1,2})^2}=\sqrt{4 + 9}=\sqrt{13}\]
    \[\tan{\varphi}=\frac{\Im(\mu_{1,2})}{\Re(\mu_{1,2})}\Rightarrow\varphi=\pi-\arctan{\frac{3}{2}}\]
    \[y_k=\rho^k(C_1\cos(\varphi k)+C_2\sin(\varphi k))\]
  \end{minipage}
\end{task}

\begin{task}
  Найдем ограниченное фундаментальное решение задачи
  \[y_{k-1}-2y_k+y_{k+1}=\sigma_k^0\]
  Так как задача неоднородная, то решение можно представить в виде
  $y_k=y_k^o+y_k^1$.
  Найдем общее решение задачи. Для этого решим $y_{k-1}-2y_k+y_{k+1}=0$
  \[P(\mu)=1-2\mu+\mu^2=0\Rightarrow\mu=1\text{ кратности 2}\Rightarrow y_k=C_1+C_2k\]
  Частное решение будем строить с помощью хитрого трюка. Запишем
  нашу задачу в виде системы
  \[\begin{cases}
      y_{k-1}-2y_k+y_{k+1}=0,\ k\leq -1                                  \\
      y_{k-1}-2y_k+y_{k+1}=1,\ k = 0 \Leftrightarrow y_{-1}-2y_0+y_{1}=1 \\
      y_{k-1}-2y_k+y_{k+1}=0,\ k \geq 1                                  \\
    \end{cases}\]

  Так как мы ищем частное решение задачи (ищем любое),
  то ради удобства занулим решение $\forall\ k\leq-1$.

  Ищем $y_0$: подставим в первое уравнение $k=-1$, тогда
  \[y_{-2}-2y_{-1}+y_{0}=0\Rightarrow y_0=0\]

  Теперь мы можем найти $y_1$, посмотрев на второе уравнение:
  \[y_{-1}-2y_0+y_{1}=1\Rightarrow y_1=1\]

  Найдем $y_2$, чтобы восстановить константы для правой части, подставив $k=1$ в третье уравнение:
  \[y_{0}-2y_1+y_{2}=0\Rightarrow y_2=2\]

  Для того чтобы найти правую часть частного решения подставим
  в известное нам общее решение:
  \[\begin{cases}
      1=C_1^++C_2^+ \\
      2=C_1^++C_2^+\cdot2
    \end{cases}\Rightarrow\begin{cases}
      C_1^+=0 \\
      C_2^+=1
    \end{cases}\Rightarrow y_k=k,\ k\geq1\]

  Таким образом искомое решение принимает вид
  \[y_k=C_1+C_2k + \begin{cases}
      0,\ k\leq 0 \\
      k,\ k\geq1
    \end{cases}= \begin{cases}
      C_1+C_2k,\ k\leq 0 \\
      C_1+(C_2+1)k,\ k\geq1
    \end{cases}\]
  $\forall\ C_1,\ C_2$ решение не будет ограниченным, ограниченных решений нет.
\end{task}

\begin{task}
  Найти все решения задачи на собственные значения:
  \[\begin{cases}
      \frac{y_{k+1}-2y_k+y_{k-1}}{h^2} = -\lambda y_k,\ h = \frac{2}{2N-1},\ 1 \leq k \leq N-1 \\
      y_0 = 0                                                                                  \\
      y_N = -y_{N-1}
    \end{cases}\]
  \begin{enumerate}
    \item Запишем канонический вид. Найдем коэффициенты для краевых условий
          \begin{align*}
            k & = 1: \frac{y_2 - 2y_1}{h^2} = -\lambda y_1                                                             \\
            k & = N-1: \frac{y_{N} - 2y_{N-1}+y_{N-2}}{h^2} = \frac{-3\cdot y_{N-1} + y_{N-2}}{h^2} = -\lambda y_{N-1}
          \end{align*}
          Таким образом задачу можно переписать в матричном виде:
          \[\underbrace{\left(\begin{array}{cccccc}
                \frac{-2}{h^2} & \frac{1}{h^2}  &               &               &                & 0              \\
                \frac{1}{h^2}  & \frac{-2}{h^2} & \frac{1}{h^2} &               &                &                \\
                               &                & \cdots        & \cdots        &                &                \\
                               &                &               & \frac{1}{h^2} & \frac{-2}{h^2} & \frac{1}{h^2}  \\
                0              &                &               &               & \frac{1}{h^2}  & \frac{-3}{h^2} \\
              \end{array}\right)}_{N-1\times N-1}
            \left(\begin{array}{c}
                y_{1}   \\
                \\
                \vdots  \\
                \\
                y_{N-1} \\
              \end{array}\right)
            =
            -\lambda
            \left(\begin{array}{c}
                y_{1}   \\
                \\
                \vdots  \\
                \\
                y_{N-1} \\
              \end{array}\right)
          \]

    \item Для более удобного решения сделаем замену $p=1-h^2\frac{\lambda}{2}$ и перепишем условие:
          \[\begin{cases}
              y_{k+1}-2py_k+y_{k-1} = 0,\ 1 \leq k \leq N-1 \\
              y_0 = 0                                       \\
              y_N = -y_{N-1}
            \end{cases}\]
          Решим получененную разностную задачу:
          \[P(\mu) = \mu^2-2p\mu+1 = 0 \Leftrightarrow \mu_{1,2}=p \pm \sqrt{p^2-1}\]
          Также по теореме Виета:
          \begin{align}
            \mu_1\cdot \mu_2      & =1  \\
            \frac{\mu_1+\mu_2}{2} & = p
          \end{align}
          \begin{enumerate}
            \item $\mu_1\neq\mu_2$. Тогда общее решение имеет вид
                  \[y_k=C_1\mu_1^k+C_2\mu_2^k\]
                  Подставим в начальные условия, чтобы найти $C_1$ и $C_2$:
                  \[\begin{cases}
                      y_0 = 0 = C_1+C_2\Rightarrow C_2=-C_1 \\
                      y_N=-y_{N-1}: C_1\mu_1^N+C_2\mu_2^N = -(C_1\mu_1^{N-1}+C_2\mu_2^{N-1})
                    \end{cases}\]
                  Преобразуем второе равенство, используя первое:
                  \[C_1(\mu_1^N-\mu_2^N)=-C_1(\mu_1^{N-1}-\mu_2^{N-1})\]
                  Если $C_1 = 0$, то $C_2=0$, то $y_k\equiv 0$, что нам неинтересно, так как нулевой вектор не является собственным. Иначе
                  \[\mu_1^N-\mu_2^N=\mu_2^{N-1}-\mu_1^{N-1} \Leftrightarrow
                    \mu_1^N+\mu_1^{N-1}=\mu_2^N+\mu_2^{N-1} \Leftrightarrow
                    \mu_1^{N-1}(\mu_1+1)=\mu_2^{N-1}(\mu_2+1)\]
                  Используем п.1 из теоремы Виета:
                  \[\mu_1^{N-1}(\mu_1+\mu_1\mu_2)=\mu_2^{N-1}(\mu_2+1)\Leftrightarrow\mu_1^N(\mu_2+1)=\mu_2^{N-1}(\mu_2+1)\]
                  Заметим, что $\mu_2\neq 1$, так как по теореме Виета $\mu_1=\mu_2=1$ -- противоречие с предположением $\mu_1\neq\mu_2$.
                  \[\frac{\mu_1^N}{\mu_2^{N-1}}=1\Leftrightarrow\mu_1^{2N-1}=1\]
                  Возьмем $2N-1$ комплексный корень из 1 и получим:
                  \[\begin{cases}
                      \mu_1^{(m)} = \exp\left(\frac{2m\pi i}{2N-1}\right)  \\
                      \mu_2^{(m)} = \exp\left(-\frac{2m\pi i}{2N-1}\right) \\
                    \end{cases} m = 0, ...,2N-2\Leftrightarrow
                    \begin{cases}
                      \mu_1^{(m)} = \exp\left(\frac{2(m-1)\pi i}{2N-1}\right)  \\
                      \mu_2^{(m)} = \exp\left(-\frac{2(m-1)\pi i}{2N-1}\right) \\
                    \end{cases} m = 1, ...,2N-1\]
                  Решение имеет вид:
                  \begin{multline*}
                    y_k=C_1\mu_1^k+C_2\mu_2^k=2C\left(\frac{\exp\left(\frac{2(m-1)\pi i}{2N-1}\right)-\exp\left(-\frac{2(m-1)\pi i}{2N-1}\right)}{2}\right) = C\sin\frac{2(m-1)\pi k}{2N-1}\\
                    m=1,...,2N-1,\ k=1,...,N-1
                  \end{multline*}
                  Найдем собственные значения. По теореме Виета:
                  \begin{multline*}
                    p=\frac{\mu_1+\mu_2}{2}=\frac{\exp\left(\frac{2(m-1)\pi i}{2N-1}\right) + \exp\left(-\frac{2(m-1)\pi i}{2N-1}\right)}{2}=\cos\left(\frac{2(m-1)\pi}{2N-1}\right)\\
                    m=1,...,2N-1
                  \end{multline*}
                  \begin{multline*}
                    p=1-\lambda\frac{h^2}{2}\Leftrightarrow\\\lambda=\frac{2}{h^2}\left(1-\cos\left(\frac{2(m-1)\pi}{2N-1}\right)\right)=\frac{4}{h^2}\sin^2\left(\frac{(m-1)\pi}{2N-1}\right)\\
                    m=1,...,2N-1
                  \end{multline*}
                  У матрицы размера $N-1\times N-1$ не может быть больше $N-1$ собственного значения, но выше мы получили $2N-1$. Отберем корни.
                  Заметим, что при $m=1$ $\lambda=0$, такое собственное значение порождает нулевой вектор, так как матрица невырождена, а нулевых
                  собственных векторов нет по определению собственного вектора. Осталось $2N-2$. Покажем,
                  что корни симметричны.

                  Обозначим $\alpha_m=\frac{\pi(m-1)}{2N-1}$. Тогда
                  \begin{align*}
                    \alpha_2      & = \frac{\pi}{2N-1}                                \\
                    \alpha_3      & = \frac{2\pi}{2N-1} = 2\alpha_1                   \\
                    ...           &                                                   \\
                    \alpha_{2N-2} & = \frac{\pi(2N-3)}{2N-1}= \pi - \frac{2\pi}{2N-1} \\
                    \alpha_{2N-1} & = \frac{\pi(2N-2)}{2N-1}= \pi - \frac{\pi}{2N-1}  \\
                  \end{align*}
                  То есть углы симметричны относительно $\frac{\pi}{2}$, а значит количество различных корней ровно $N-1$.

            \item $\mu_1 = \mu_2$: из теоремы Виета следует
                  $\mu_1=\mu_2=p$ и $\mu_1\mu_2 = 1$ $\Rightarrow$ $\lambda=0$ и $\lambda=\frac{4}{h^2}$.
                  Вставим это решение в ответ из случая $\mu_1\neq\mu_2$
          \end{enumerate}
          Итоговый ответ:
          \begin{align*}
            y_k     & = C\sin\frac{2m\pi k}{2N-1}                         & m=1,...,N-1 \\
            \lambda & = \frac{4}{h^2}\sin^2\left(\frac{m\pi}{2N-1}\right) & k=0,...,N
          \end{align*}
  \end{enumerate}
\end{task}

\begin{task}
\end{task}

\begin{task}
  Среди всех многочленов вида $5x^3+a_2x^2+a_1x+a_0$
  найти наимее уклоняющийся от нуля на отрезке $[1,2]$.

  \begin{enumerate}
    \item В классе многочленов степени $n$ удовлетворяющих
          условию $P_n^{(k)}(0)=c\cdot k!\neq0$
          наименее уклоноящийся от 0 на $[a,b]$ имеет вид:
          \[P_n^*(x)=ck!\left(\frac{b-a}{2}\right)^k\frac{T_n\left(\frac{2x-(a+b)}{b-a}\right)}{T_n^{(k)}\left(\frac{a+b}{a-b}\right)}\]
          В условиях нашей задачи $c=5,\ k=3,\ n=3,\ a=1,\ b=2$
          \[T_0=1,\ T_1=x,\ T_2=2x^2-1,\ T_3=2x(2x^2-1)-x=4x^3-3x,\ T_3^{(3)}=4\cdot3!\]
          \[P_n^*(x)=5\cdot 3!\left(\frac{1}{2}\right)^3\frac{T_3\left(2x-3\right)}{4\cdot3!}=\frac{5}{32} T_3(2x-3)\]
    \item Докажем, что такой многочлен действительно наименее уклоняющийся.

          Пусть $\exists \tilde{P}_n^*:\ \norm{\tilde{P}_n^*}<\norm{P_n^*}$. Рассмотрим $Q_n=\tilde{P}_n^*-P_n^*$
          \[\sgn Q_n|_{x_{(m)}}=(-1)^m,\ m=0,\ldots,n\]
          То есть $Q_n$ имеет $n+1$ экстремум, то есть $n$ различных корней.
          Значит $Q_n^{(k)}$ имеет $n-k$ корней на $[a,b]$, но помимо них есть еще корень $0$,
          так как коэффициенты при $x^k$ совпадают, так как $\tilde{P}_n^*,\ P_n^*$ из одного класса.
          Таким образом $Q_n^{(k)}\equiv 0\Rightarrow Q_n$ - многочлен $k-1$ степени, но
          корней у него $n$, значит $Q_n\equiv0$.
  \end{enumerate}

  Ответ: $\frac{5}{32} T_3(2x-3)=\frac{5}{32}(32x^3-144x^2+210x-99)$
\end{task}

\begin{task}
\end{task}

\begin{task}
\end{task}

\begin{task}
\end{task}

\begin{task}
\end{task}

\begin{task}
\end{task}

\begin{task}
\end{task}

\begin{task}
\end{task}

\begin{task}
\end{task}

\begin{task}
  Для задачи
  \[-y''(x) +2y(x) = f(x),\ y(0) = 1,\ y'(1) = 0 \]
  Построить разностную схему второго порядка аппроксимации на
  сетке $x_i=ih,\ i=0,\ldots,N,\ h=\frac{1}{N}$. Исследовать устойчивость.

  В качестве разностного уравнения для разностной схемы предлагается брать
  \[-\frac{y_{k+1}-2y_k+y_{k-1}}{h^2}+2y_k = f_k,\ h = \frac{1}{N},\ 1 \leq k \leq N-1\]
  Такое уравнение обеспечит второй порядок аппроксимации, что мы проверим позже.
  Осталось подобрать краевые условия для схемы: для $y(0)=1$ обеспечит аппроксимацию
  любого порядка точное значение $y_0=1$, тогда как для $y'(1)=0$ нужно
  подобрать что-то особенное: знаем, что $\frac{y_k-y_{k-1}}{h}$ дает
  первый порядок аппроксимации, предлагается использовать $\delta$-поправку,
  чтобы получить второй. Будем искать дополнительное слагаемое из условия аппроксимации
  \begin{multline*}
    \abs{\frac{y(1)+y(1-h)}{h}-\delta}=\abs{\frac{y(1)+y(1)-hy'(1)+\frac{h^2}{2}y''(1)+\bigO(h^3)}{h}-\delta}=\abs{\frac{h}{2}y''(1)+\bigO(h^2)-\delta}\leq ch^2\Leftrightarrow\delta=\frac{h}{2}y''(1)= \\
    \frac{h}{2}(2y(1)-f(1))
  \end{multline*}
  Тогда краевое условие будет иметь вид
  \[\frac{y_N-y_{N-1}}{h}+\frac{h}{2}(2y_N-f_N)=0\Leftrightarrow 2\frac{y_N-y_{N-1}}{h^2}+2y_N=f_N\]
  Итоговая разностная схема имеет вид
  \[\begin{cases}
      -\frac{y_{k+1}-2y_k+y_{k-1}}{h^2}+2y_k = f_k,\ h = \frac{1}{N},\ 1 \leq k \leq N-1 \\
      x_k = kh,\ f_k := f(x_k),\ p_k := p(x_k)                                           \\
      y_0 = 1                                                                            \\
      2\frac{y_N-y_{N-1}}{h^2}+2y_N=f_N
    \end{cases}\]
  \[k=1: \frac{y_{2}-2y_1+1}{h^2}+2y_1 = f_1\]
  Задача в матричном виде имеет вид:
  \[
    -\left(\begin{array}{cccccc}
        \frac{-2}{h^2} & \frac{1}{h^2}  &               &               &                & 0              \\
        \frac{1}{h^2}  & \frac{-2}{h^2} & \frac{1}{h^2} &               &                &                \\
                       &                & \cdots        & \cdots        &                &                \\
                       &                &               & \frac{1}{h^2} & \frac{-2}{h^2} & \frac{1}{h^2}  \\
        0              &                &               &               & \frac{2}{h^2}  & \frac{-2}{h^2} \\
      \end{array}\right)
    \left(\begin{array}{c}
        y_{1}  \\
        \\
        \vdots \\
        \\
        y_{N}  \\
      \end{array}\right)
    +
    2
    \left(\begin{array}{c}
        y_{1}  \\
        \\
        \vdots \\
        \\
        y_{N}  \\
      \end{array}\right)
    =
    \left(\begin{array}{c}
        f_{1}  \\
        \\
        \vdots \\
        \\
        f_{N}  \\
      \end{array}\right)
  \]
  Будем использовать сеточную интегральную норму
  \[\Vert y_h\Vert^2_h = (y_h,y_h)_h= \sum_{k=1}^{N-1}y^2_kh\]
  согласованной с нормой $\Vert y(x)\Vert^2_{L^2(0,1)} = \int_0^1y^2(x)dx$ исходной задачи.

  План:
  \begin{enumerate}[I.]
    \item Доказать, что разностная схема имеет порядок аппроксимации $O(h^2)$.
    \item Доказать устойчивость разностный схемы.
    \item Доказать, что есть сходимость $O(h^2)$.
  \end{enumerate}

  \newpage

  \begin{enumerate}[I.]
    \item Докажем аппроксимацию второго порядка на решении:
          \begin{enumerate}
            \item $\Vert L_h(y)_{Y_h}-f_h\Vert_{F_h}\leq\bigO(h^2)$
                  \[\max_{x_k}\left|-\frac{y(x_k+h)-2y(x_k)+y(x_k-h)}{h^2}+2y(x_k)-f(x_k)\right|=\]
                  \[y(x_k\pm h)=y(x_k)\pm hy'(x_k)+\frac{h^2}{2}y''(x_k)\pm\frac{h^3}{6}y'''(x_k)+\bigO(h^4)\]
                  \[=\max_{x_k}\left|-\frac{h^2y''(x_k)+\bigO(h^4)}{h^2}+2y(x_k)-f(x_k)\right|=\]
                  \[=\max_{x_k}\left|-y''(x_k)+\bigO(h^2)+2y(x_k)-f(x_k)\right|=\bigO(h^2)\]
            \item $\Vert l_h(y)_{Y_h}-\varphi_h\Vert_{\Phi_h}\leq\bigO(h^2)$
                  \[y_0 = 1:\ \Vert y(0)-1\Vert_{\Phi_h}=0\]
                  \begin{multline*}
                    \frac{y_N - y_{N-2}}{2h} = 0:\ \norm{\frac{y(1)-y(1-2h)}{2h}}_{\Phi_h}= \\
                    y\left(1-2h\right)=y(1)-2hy'(1)+\frac{4h^2}{2}y''(x_k)+\bigO(h^3) \\
                    =\norm{-y'(1)+2hy''(x_k)+\bigO(h^2)}_{\Phi_h}=\bigO(h^2)
                  \end{multline*}
            \item Условия нормировки
                  \begin{align*}
                    \lim_{h\rightarrow0}\Vert f_h-(f)_{F_h}\Vert_{F_h}                   & =0 \Rightarrow f(x_k)-f(x_k) = 0       \\
                    \lim_{h\rightarrow0}\Vert \varphi_h-(\varphi)_{\Phi_h}\Vert_{\Phi_h} & =0 \Rightarrow (0\ 0)^T - (0\ 0)^T = 0
                  \end{align*}
          \end{enumerate}
          Значит схема имеет \textbf{второй порядок аппроксимации}.

          \textbf{Замечание}: Доказали аппроксимацию на решении в $\Vert\cdot\Vert_{\infty}$, но
          \[\Vert x\Vert_h=\sqrt{\sum_{i=1}^{N-1}x_i^2h}\leq\max_i|x_i|\sqrt{\sum_{i=1}^{N-1}h}\leq\max_i|x_i|\sqrt{\frac{2(N-1)}{2N-1}}\leq\max_i|x_i|\cdot1=\Vert x\Vert_{\infty}\]
          То есть из аппроксимации в $\Vert\cdot\Vert_{\infty}$ следует аппроксимация в $\Vert\cdot\Vert_h$.

          \newpage

    \item Напомним определение устойчивости разностной схемы.
          \begin{definition}
            Пусть уравнение $y''(x)=f(x)$ доопределено краевыми
            условиями на разных концах отрезка. Разностная схема
            $A_hy_h = f_h$ линейной задачи устойчива, если существуют $C$, $h_0$ такие, что для
            произвольных $A_hy^{(1,2)}_h = f^{(1,2)}_h$ выполняется оценка
            \[\Vert y^{(1)}_h-y^{(2)}_h\Vert_h\leq C\Vert f^{(1)}_h -f^{(2)}_h\Vert_h\ \forall h\leq h_0\]
            с константой $C$, не зависящей от $h$.
          \end{definition}

          Будем доказывать устойчивость разностной схемы энергетическим методом.
          Запишем нашу дифференциалную задачу
          \[-y''(x)+p(x)y(x)=f(x),\ y(0) = y'(1) = 0,\ p(x)\geq 0\]
          Умножим уравнение на $y(x)$, и результат проинтегрируем по отрезку $[0, 1]$
          \[\int_0^1 (-y''y+py^2)dx = \int_0^1fydx \]
          \[\int_0^1 -y''ydx+ \int_0^1py^2 dx = \int_0^1fydx \]
          Проинтегрируем по частям первое слагаемое
          \[\int_0^1 -y''ydx = \int_0^1-ydy' = -yy'\vert^1_0 - \int_0^1y'd(-y) = \int_0^1(y')^2dx\]
          Получили интегральное тождество
          \[\int_0^1 (y'(x))^2dx+ \int_0^1py^2 dx = \int_0^1fydx \]
          Оценим слева через неравенство, связывающее интегралы от квадратов
          функции и ее производной. Так как $y(0) = 0$, то справедливо следующее:
          \[y(x_0) = \int_0^{x_0}y'(x)dx\]
          Применим интегральную форму неравенства Коши-Буняковского:
          \[|y(x_0)|^2 = \left|\int_0^{x_0}y'dx\right|^2\leq\left(\int_0^{x_0}1^2dx\right)\left(\int_0^{x_0}(y')^2dx\right)\leq\int_0^{x_0}(y')^2dx\leq\int_0^{1}(y')^2dx\]
          После интегрирования по $x_0$ по отрезку $[0,1]$ обеих частей получим искомое равенство
          \[\int_0^1|y(x_0)|^2dx_0 \leq \int_0^{1}(y')^2dx\int_0^1dx_0 \Leftrightarrow \int_0^1y^2dx\leq\int_0^1(y')^2dx\]
          Оценку справа выведем из разности квадратов:
          \[0\leq\int_0^1(f - y)^2dx\leq\int_0^1f^2dx-2\int_0^1fydx+\int_0^1y^2dx\]
          \[\Rightarrow\int_0^1fydx\leq\frac{1}{2}\left(\int_0^1f^2dx + \int_0^1y^2dx\right)\]

          Таким образом имеем:
          \[\int_0^1y^2dx\leq\int_0^1 (y'(x))^2dx+ \int_0^1py^2 dx = \int_0^1fydx\leq\frac{1}{2}\left(\int_0^1f^2dx + \int_0^1y^2dx\right)\]
          Таким образом верна оценка
          \[\int_0^1y^2dx\leq\int_0^1f^2dx\Rightarrow\Vert y\Vert_{L_2(0,1)}\leq\Vert f\Vert_{L_2(0,1)}\]
          Это означает устойчивость дифференциальной задачи по правой части.

          Докажем теперь устойчивость разностной схемы.
          \[-\frac{y_{k+1}-2y_k+y_{k-1}}{h^2}+p_ky_k = f_k,\ 1 \leq k \leq N-1,\ y_0 = 0,\ y_N = y_{N-1}\]
          Умножим на $y_k$ и просуммируем от $1$ до $N-1$. Так как $y_0 = 0,\ y_N = y_{N-1}$
          \[-\frac{1}{h^2}\left(\sum_{k=1}^{N-1}\left(y_{k+1}-2y_k+y_{k-1}\right)y_k\right)=-\frac{1}{h^2}\left(\sum_{k=1}^{N-1}\left(y_{k+1}-y_k-y_k+y_{k-1}\right)y_k\right)=\]
          \[=-\frac{1}{h^2}\sum_{k=1}^{N-1}\left(y_{k+1}-y_k\right)y_k+\frac{1}{h^2}\sum_{k=1}^{N-1}\left(y_k-y_{k-1}\right)y_k=\]
          \[=-\frac{1}{h^2}\sum_{k=2}^{N}\left(y_{k}-y_{k-1}\right)y_{k-1}+\frac{1}{h^2}\sum_{k=1}^{N-1}\left(y_k-y_{k-1}\right)y_k=\frac{1}{h^2}\sum_{k=1}^{N}(y_k-y_{k-1})^2\]
          Получили конечномерный аналог интегрального тождества:
          \[\frac{1}{h^2}\sum_{k=1}^N(y_k-y_{k-1})^2+\sum_{k=1}^{N-1}p_ky_k^2=\sum_{k=1}^{N-1}f_ky_k\]
          Для оценки слева докажем сеточный аналог неравенства для функции и ее производной в точках $k=1,...,N-1$.
          Так как $y_0 = 0$, справедливо следующее:
          \[y_k=\sum_{i=1}^{k}(y_i-y_{i-1})\]
          Воспользуемся неравенством Коши-Буняковского и $y_N=y_{N-1}$
          \[y_k^2\leq\left(\sum_{i=1}^k1^2\right)\left(\sum_{i=1}^k(y_i-y_{i-1})^2\right)\leq (N-1)\sum_{i=1}^{N-1}(y_i-y_{i-1})^2\]
          Суммируя до $N-1$ обе части, при $h=\frac{2}{2N-1}$ получаем оценку:
          \[\sum_{k=1}^{N-1}y_k^2\leq(N-1)^2\sum_{k=1}^{N-1}(y_k-y_{k-1})^2\leq\frac{1}{h^2}\sum_{k=1}^{N-1}(y_k-y_{k-1})^2\]
          Найдем аналогично дифференциальному неравенству оценку справа
          \[0\leq\sum_{k=1}^{N-1}(f_k-y_k)^2 = \sum_{k=1}^{N-1}f_k^2-2\sum_{k=1}^{N-1}f_ky_k+\sum_{k=1}^{N-1}y_k^2\]
          \[\Rightarrow\sum_{k=1}^{N-1}f_ky_k\leq\frac{1}{2}\left(\sum_{k=1}^{N-1}f_k^2+\sum_{k=1}^{N-1}y_k^2\right)\]
          Итоговая оценка имеет вид
          \[\sum_{k=1}^{N-1}y_k^2\leq\frac{1}{h^2}\sum_{k=1}^{N-1}(y_k-y_{k-1})^2+\sum_{k=1}^{N-1}p_ky_k^2=\sum_{k=1}^{N-1}f_ky_k\leq\frac{1}{2}\left(\sum_{k=1}^{N-1}f_k^2+\sum_{k=1}^{N-1}y_k^2\right)\]
          Таким образом
          \[\sum_{k=1}^{N-1}y_k^2\leq\sum_{k=1}^{N-1}f_k^2\Rightarrow\sum_{k=1}^{N-1}y_k^2h\leq\sum_{k=1}^{N-1}f_k^2h\Rightarrow\Vert y_h\Vert^2_h\leq\Vert f_h\Vert^2_h\]
          То есть \textbf{разностная схема устойчива} в норме $\Vert\cdot\Vert_h$.
          \newpage
    \item Докажем, что у схемы есть сходимость порядка $\bigO(h^2)$.
          \begin{theorem}[Филиппов А.Ф.]
            Пусть выполнены следюущие условия:
            \begin{enumerate}
              \item операторы $L$, $l$ и $L^h$, $l^h$ - линейные;
              \item решение дифференциальной задачи $\exists!$;
              \item разностная схема аппроксимирует на решении дифференциальную задачу с порядком $p$;
              \item разностная схема устойчива;
            \end{enumerate}
            Тогда решение разностной схемы сходится к решению дифференциальной задачи с порядком не ниже $p$
          \end{theorem}
          Посмотрим на наши результаты
          \begin{enumerate}
            \item операторы $L$, $l$ и $L^h$, $l^h$ - действительно линейные;
            \item решение дифференциальной задачи $\exists!$, так как по условию $y$ и $f$ хорошие гладкие функции.
            \item разностная схема аппроксимирует на решении дифференциальную задачу с порядком $2$;
            \item разностная схема действительно устойчива;
          \end{enumerate}
          Таким образом решение разностной схемы \textbf{сходится} к решению дифференциальной задачи \textbf{с порядком не ниже $2$}.
  \end{enumerate}
\end{task}


\end{document}
