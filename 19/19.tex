\section{Устойчивость краевой задачи для уравнения второго порядка: метод собственных функций.}

Напомним определение устойчивости
\begin{definition}
  Разностная схема $\begin{cases}
      L_hy_h = f_h \\
      l_hy_h = \varphi_h
    \end{cases}$
  называется устойчивой, если:
  $\forall y_h^{(1)}$, $y_h^{(2)}\ \forall \varepsilon > 0\ \exists\delta=\delta(\varepsilon):$
  \[\norm{f^{(1)}_h-f^{(2)}_h}+\norm{\varphi^{(1)}_h-\varphi^{(2)}_h}\leq\delta,\ \forall h\leq h_0\Rightarrow\norm{y^{(1)}_h-y^{(2)}_h}\leq\varepsilon\]
\end{definition}
\begin{definition}
  Линейная схема $\begin{cases}
      L_hy_h = f_h \\
      l_hy_h = \varphi_h
    \end{cases}$
  называется устойчивой, если:
  \[\norm{y^{(1)}_h-y^{(2)}_h}\leq C\left(\norm{f^{(1)}_h-f^{(2)}_h}+\norm{\varphi^{(1)}_h-\varphi^{(2)}_h}\right),\ \forall h\leq h_0\]
  $C$ не должна зависеть от $h$.
\end{definition}
Если в разностной схеме матрица $L_h$ является не вырожденной, то $y_h=L_h^{-1}f_h$.
Отсюда следует неравенство для нормы векторов
\[\norm{y_h^{(1)}-y_h^{(2)}}_h\leq\norm{L^{-1}_h}_h\norm{f_h^{(1)}-f_h^{(2)}}_h\]
То есть, чтобы удостовериться в том, что схема устойчива надо выбрать $C\geq \norm{L_h^{-1}}_h$.

Исследуем устойчивость разностной схемы
\begin{equation}\label{eq:scheme}
  -\frac{y_{i+1}-2y_i+y_{i-1}}{h^2}=f_k,\ y_0=y_N=0,\ h=1/N \Leftrightarrow L_hy_h=f_h
\end{equation}

в сеточной интегральной норме $\norm{y_h}^2_h=(y_h,y_h)_h=\sum_{k=1}^{N-1}y_k^2$, согласованной с непрерывной нормой $\norm{y(x)}_{L_2(0,1)}^2=\int_{0}^{1}y^2(x)dx$ исходной задачи.
Будем оценивать норму оператора $\norm{L_h^{-1}}_h$, где
\[\norm{L_h^{-1}}_h^2\defeq\sup_{y_h\neq0}\frac{\norm{L_h^{-1}y_h}_h}{\norm{y_h}_h}=\sup_{y_h\neq0}\frac{(L_h^{-1}y_h,L_h^{-1}y_h)_h}{(y_h,y_h)_h}\]
Пусть известны собственные числа $\lambda_n$ матрицы $L_h$,
собственные вектора ортонормальны: $(y^{(n)},y^{(m)})=\delta_m^n$.
Тогда
\[\norm{L_h^{-1}}_h^2=\sup_{\{c_n\}}\frac{(\sum\lambda^{-1}_nc_ny_h^{(n)},\sum\lambda^{-1}_nc_ny_h^{(n)})_h}{(\sum c_ny_h^{(n)},\sum c_ny_h^{(n)})_h}=\sup_{\{c_n\}}\frac{\sum\lambda^{-2}_nc^2_n}{\sum c_n^2}=\max_n{\abs{\lambda_n^{-2}}}\sup_{\{c_n\}}\frac{\sum c^2_n}{\sum c_n^2}=\max_n{\abs{\lambda_n^{-2}}}\]
То есть $\norm{L_h}_h=\lambda^{-1}_{\min}$. Решив разностную задачу $\eqref{eq:scheme}$ аналитически, получим
\[y_k^{(n)}=\sin\pi nkh,\ \lambda_n=\frac{4}{h^2}\sin^2\frac{\pi nh}{2},\ n=1,\ldots,N-1\]
Проверим, что получившиеся решения уравнения являются ортогональными векторами.
Матрица $L_h$ симметрична $L_h=L_h^T$, а это значит, что
\[(L_hy^{(n)},y^{(m)})_h=(y^{(n)},L_hy^{(m)})_h\Leftrightarrow 0=(L_hy^{(n)},y^{(m)})_h-(y^{(n)},L_hy^{(m)})_h=(\lambda^{(n)}-\lambda^{(m)})(y^{(n)},y^{(m)})_h\]
То есть $(y^{(n)},y^{(m)})_h=0$ для $\lambda^{(m)}\neq\lambda^{(n)}$.
Ортогональность собственных векторов доказана.

\begin{minipage}{.5\linewidth}
  \tikzsetnextfilename{19/SinInequation}
  \begin{tikzpicture}[scale=1.5]
    \draw (-2,0) grid (2,2);
    \draw [domain=-2:2,color=red] plot function{abs(sin(x))} node[anchor=north west]{$\sin\abs{x}$};
    \draw [domain=-2:2,color=blue] plot function{abs(x*2/pi)} node[right]{$\frac{2}{\pi}\abs{x}$};
  \end{tikzpicture}
\end{minipage}\hfill
\begin{minipage}{.5\linewidth}
  Из неравенства $\sin\abs{x}\geq\frac{2}{\pi}\abs{x}$ при $\abs{x}\leq\frac{\pi}{2}$ имеем
  \[\lambda_{\min}=\lambda_1=\frac{4}{h^2}\sin^2\frac{\pi h}{2}\geq\frac{4}{h^2}h^2\geq4\]
  \[\lambda_{\max}=\lambda_{N-1}=\frac{4}{h^2}\sin^2\frac{\pi h(N-1)}{2}\leq\frac{4}{h^2}\sin^2\frac{\pi}{2}\leq\frac{4}{h^2}\]
\end{minipage}
Таким образом, верна не зависящая от $h$ оценка для нормы $\norm{L_h^{-1}}_h\leq\frac{1}{4}$,
то есть мы доказали устойчивость схемы по определению.
