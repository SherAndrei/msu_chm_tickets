\section{Решение задачи наименьших квадратов полного и неполного рангов методом сингулярного разложения.}

\begin{definition}
  Пусть $A\in\R^{m\times n}$, $m\geq n$. Тогда справедливо \textit{сингулярное} разложение
  \[A=U\Sigma V^T=(U_1U_2)\left(\begin{array}{c}
        \Sigma_n \\ 0
      \end{array}\right)V^T\]
  \begin{itemize}
    \item $U\in\R^{m\times m}=(\mathbf{u}_1,\ldots,\mathbf{u}_m)$ - ортогональная матрица \textit{левых сингулярных векторов}, $U_1\in\R^{m\times n}$, $U_2\in\R^{m\times (m-n)}$
    \item $V\in\R^{n\times n}=(\mathbf{v}_1,\ldots,\mathbf{v}_n)$ - ортогональная матрица \textit{правых сингулярных векторов}.
    \item $\Sigma_n\in\R^{n\times n}$ - диагональная матрица,
          на диагонали которой расположены упорядоченные \textit{сингулярные числа} $\sigma_1\geq\sigma_2\geq\ldots\sigma_n\geq0$.
  \end{itemize}
\end{definition}

\begin{theorem}
  Пусть $A=U\Sigma V^T$, $\text{rk}(A)=n$. Тогда решением ЗНК является вектор
  \[\mathbf{x}=V\Sigma_n^{-1}U_1^T\mathbf{b}\]
\end{theorem}
\begin{proof}
  Так как искомый вектор является решением нормального уравнения, то
  \[A^TA\mathbf{x}=A^T\mathbf{b}\Leftrightarrow \mathbf{x}=(A^TA)^{-1}A^T\mathbf{b}\Leftrightarrow \mathbf{x}=(V\Sigma_n^2V^T)^{-1}V^T\Sigma^T U\mathbf{b}=V\Sigma_n^{-2}\Sigma_n^T U_1\mathbf{b}=V\Sigma_n^{-1}U_1\mathbf{b}\]
\end{proof}

\begin{theorem}
  Пусть
  \[A=(U_1U_2)\left(\begin{array}{cc}
        \Sigma_k & 0 \\
        0        & 0
      \end{array}\right)
    \left(\begin{array}{c}
        V_1^T \\
        V_2^T
      \end{array}\right),\ \text{rk}(A)=k<n<m\]
  Здесь $U_1\in\R^{m\times k}$, $\Sigma_k\in\R^{k\times k}$, $V_1\in\R^{n\times k}$.
  Тогда решением ЗНК является пространство решений размерности $n-k$ вида
  \[\mathbf{x}=V_1\Sigma_k^{-1}U_1^T\mathbf{b}+V_2\mathbf{z},\ \mathbf{z}\in\R^{n-k}\]
\end{theorem}
\begin{proof}
  Исходную задачу домножим на $U^T$. Итоговая норма
  не изменится, так как $U^T$ является ортогональной матрицей, то есть сохраняет длину.
  \[\inf_{\tilde{\mathbf{x}}}\norm{\mathbf{b}-A\tilde{\mathbf{x}}}_2^2
    =\inf_{\tilde{\mathbf{x}}}\norm{U^T\mathbf{b}-\Sigma V^T\tilde{\mathbf{x}}}_2^2
    =\inf_{\tilde{\mathbf{x}}}\norm{
      \left(\begin{array}{c}
          U_1^T \\
          U_2^T\end{array}\right)
      \mathbf{b}-\left(\begin{array}{cc}
          \Sigma_{k} & 0 \\
          0          & 0\end{array}\right)
      \left(\begin{array}{c}
          V_1^T \\
          V_2^T
        \end{array}\right)
      \left(\begin{array}{c}
          \tilde{\mathbf{x}}_1 \\
          \tilde{\mathbf{x}}_2\end{array}\right)
    }_2^2\]
  \[=\inf_{\tilde{\mathbf{x}}}\norm{
      \begin{array}{c}
        U_1^T\mathbf{b}-\Sigma_k V_1^T\tilde{\mathbf{x}}_1 \\
        U_2^T\mathbf{b}\end{array}
    }_2^2
    =\inf_{\tilde{\mathbf{x}}}\left(\norm{U_1^T\mathbf{b}-\Sigma_k V_1^T\tilde{\mathbf{x}}_1}_2^2 + \norm{U_2^T\mathbf{b}}^2_2\right)\]
  Таким образом, минимум достигается только при
  \[U_1^T\mathbf{b}-\Sigma_k V_1^T\mathbf{x}_1=0\Leftrightarrow\mathbf{x}_1=V_1\Sigma_k^{-1}U_1^T\mathbf{b}\]
  Найдем общий вид решения. Так как $\dim\ker(A)=n-k$ и $V_1^TV_2=0$ вследствие ортогональности $\mathbf{v}_i$,
  то $U_1^T\mathbf{b}-\Sigma_k V_1^T(\mathbf{x}_1+V_2\mathbf{z})=0$. Значит общее решение имеет вид
  \[\mathbf{x}=V_1\Sigma_k^{-1}U_1^T\mathbf{b}+V_2\mathbf{z},\ \mathbf{z}\in\R^{n-k}\]
\end{proof}
\begin{remark}
  $SVD$-разложение в отличие от $QR$-разложения отвечает на вопрос
  какой вектор в пространстве решений $\R^{n-k}$ имеет минимальную
  норму. Действительно,
  \[(\mathbf{x},\mathbf{x})=(\mathbf{x}_1,\mathbf{x}_1)+2(\mathbf{x}_1,V_2\mathbf{z})+(V_2\mathbf{z},V_2\mathbf{z})=(\mathbf{x}_1,\mathbf{x}_1)+(\mathbf{z},\mathbf{z})\]
  при $\mathbf{z}=0$ достигается минимум.
\end{remark}
