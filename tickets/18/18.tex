\section[Обыкновенные дифференциальные уравнения второго порядка, аппроксимация, alpha-устойчивость. Аппроксимация краевых условий третьего рода.]{Обыкновенные дифференциальные уравнения второго порядка, аппроксимация, $\alpha$-устойчивость. Аппроксимация краевых условий третьего рода.}

\subsection*{Обыкновенные дифференциальные уравнения второго порядка}

\begin{definition}
  Будем рассматривать только задачи с правыми
  частями, не зависящими от $y'$: $f(x,y,y')=f(x,y)$.
  вида
  \begin{equation}\label{eq:second_cauchy:task}
    y''=f(x,y)
  \end{equation}
  Для такой задачи предлагается строить следующую разностную схему
  \begin{equation}\label{eq:second_cauchy:recur_scheme}
    \begin{array}{c}
      h=\frac{X}{N},\ x_i=x_o+i\cdot h,\ y_h=\{y_i\}_{i=0}^{N},\ \norm{y_h}_{Y_h}=\max\limits_{i}\abs{y_k} \\
      \begin{cases}
        \frac{1}{h^2}\sum_{i=0}^na_{-i}y_{k-i}=\sum_{i=0}^nb_{-i}f_{k-i},\ k=n,n+1\ldots \\
        y_0,y_1,\ldots,y_{n-1} - \text{ начальные условия}
      \end{cases}                                                                            \\
    \end{array}
  \end{equation}
  где $a_{-i}$, $b_{-i}$ не зависят от $h$, $a_0$, $a_n$ $\neq0$ и $f_{k-i}=f(x_{k-i},y_{k-i})$.
\end{definition}

Хотим для такой разностной схемы проверять условие аппроксимации на решении,
чтобы далее пользоваться теоремой Филиппова. Зададим функцию погрешности
\[r^k_h\defeq\frac{1}{h^2}\sum_{i=0}^na_{-i}y(x_{k-i})-\sum_{i=0}^nb_{-i}f(x_{k-i},y(x_{k-i}))\]

Условия аппроксимации на решении с порядком $p$ на отрезке $[x_0,x_0+X]$:
$\begin{cases}\norm{r_h}_{F_h}\leq ch^p \\
    \norm{f_h-(f)_h}_{F_h}\underset{h\rightarrow0}{\rightarrow}0
  \end{cases}$

Коэффициенты в общем случае $a_{-i}$ и $b_{-i}$ находятся из условий
аппроксимации на задаче
\[\norm{L_h(y)_{Y_h} - (Ly)_{F_h}}_{F_h}+\norm{(f)_{F_h}-f_h}_{F_h}\leq c_1h^{p_1}\]

\begin{theorem}[Необходимые и достаточные условия аппроксимации на решении]
  Для задачи \eqref{eq:second_cauchy:task} с разностной схемой \eqref{eq:second_cauchy:recur_scheme}
  необходимые и достаточные условия аппроксимации на решении имеют вид
  \[\sum_{i=0}^na_{-i}=0;\ \sum_{i=0}^nb_{-i}=1;\ \sum_{i=0}^na_{-i}i=0;\sum_{i=0}^ni^2a_{-i}=2\]
\end{theorem}
\begin{proof}
  \begin{enumerate}
    \item Проверим условие нормировки правых частей
          \[\norm{f_h-(f)_h}_{F_h}=\norm{f(x_{k-i})-\sum_{i=0}^nb_{-i}f(x_{k-i})}_{F_h}=\norm{f(x_{k-i})\left(1-\sum_{i=0}^nb_{-i}\right)}_{F_h}\underset{h\rightarrow0}{\rightarrow}0\Leftrightarrow\sum_{i=0}^nb_{-i}=1\]
    \item Выпишем ряд Тейлора для левой и правой части в узлах $\{x_k\}$:
          \begin{align*}
            y(x_k-ih)=y(x_k)-ihy'(x_k)+\frac{(ih)^2}{2}y''(x_k) + \bigO(h^3) \\
            f(x_k-ih)=y''(x_k-ih)=y''(x_k)-ihy'''(x_k) + \bigO(h^2)          \\
          \end{align*}
          Запишем условие аппроксимации на решении:
          \begin{multline*}
            \abs{\frac{1}{h^2}\sum_{i=0}^na_{-i}y_{k-i}-\sum_{i=0}^nb_{-i}f(x_k-ih)} \\
            = \left|\frac{1}{h^2}\sum_{i=0}^na_{-i}y(x_k)+\frac{1}{h^2}\sum_{i=0}^na_{-i}(-ihy'(x_k))+\frac{1}{h^2}\sum_{i=0}^na_{-i}\left(\frac{(ih)^2}{2}y''(x_k)\right)
            -\sum_{i=0}^nb_{-i}y''(x_k)+\bigO(h)\right|\leq
          \end{multline*}
          \[ \leq\abs{\frac{y(x_k)}{h^2}\sum_{i=0}^na_{-i}+\frac{y'(x_k)}{h}\left(-\sum_{i=0}^na_{-i}i\right)+y''(x_k)\left(\sum_{i=0}^na_{-i}\frac{i^2}{2}-\sum_{i=0}^nb_{-i}\right)+\bigO(h)}\leq ch \]
          Для выполнения условия аппроксимации нужно, чтобы все коэффициенты до $h$ занулились, отсюда
          и из проверки нормировки следует доказательство теоремы.
  \end{enumerate}
\end{proof}
\begin{remark}
  Для того, чтобы найти $a_{-i}$ и $b_{-i}$ надо решить систему уравнений.
  Для того, чтобы система была разрешима, важно проверить $p=2n$.
\end{remark}

\subsection*{$\alpha$-устойчивость задачи второго порядка}

В случае задачи \eqref{eq:second_cauchy:task} проверяют
более слабое определение $\alpha$-устойчивости.

\begin{definition}
  Для задачи Коши $y''=f(x)$ схема называется $\alpha$-
  устойчивой, если все корни соответствующего характеристического
  многочлена $\sum_{i=0}^na_{-i}\mu^{k-i}$ однородного уравнения
  принадлежат единичному кругу и на границе круга нет кратных корней, за
  исключением $\mu=1$ кратности $2$.
\end{definition}

Отличие условия устойчивости для задачи Коши второго порядка
от условия устойчивости для задачи первого порядка обусловлено
более высокой степенью $h$ в правой части разностной схемы $\sum_{i=0}^na_{-i}y_{k-i}=h^2\sum_{i=0}^nb_{-i}f_{k-i}$.

\subsection*{Аппроксимация граничных условий третьего рода}
Пусть дана задача
\[\begin{cases}
    -(k(x)y')'+p(x)y=f(x),\ 0<k_0<k(x)<k_1,\ 0\leq p(x)\leq p_1 \\
    ay+by'=c
  \end{cases}\]
Предлагается выбрать следующую разностную схему
\[-\frac{1}{h}\left[k(x_{i+1/2})\frac{y_{i+1}-y_i}{h}-k(x_{i-1/2})\frac{y_{i}-y_{i-1}}{h}\right]+p(x_i)y_i=f_i\]
Проверим, что она второго порядка аппроксимации на решении:
\[\abs{-\frac{1}{h}\underbrace{\left[k\left(x_k+\frac{h}{2}\right)\frac{y(x_k+h)-y(x_k)}{h}-k\left(x_k-\frac{h}{2}\right)\frac{y(x_k)-y(x_k-h)}{h}\right]}_{\star}+p(x_k)y(x_k)-f(x_k)}\leq ch^2\]
Напомним формулу разложения в ряд Тейлора в точке $x_k$
\[u(x\pm h)=u(x)\pm hu'(x)+\frac{h^2}{2}u''(x)+\bigO(h^3)\]
Отдельно решим то, что помечено $\star$:
\begin{multline*}
  \left(k(x_k)+\frac{h}{2}k'(x_k)+\frac{h^2}{8}k''(x_k)+\bigO(h^3)\right)\left(y'(x_k)+\frac{h}{2}k''(x_k)+\bigO(h^2)\right)- \\
  \left(k(x_k)-\frac{h}{2}k'(x_k)+\frac{h^2}{8}k''(x_k)+\bigO(h^3)\right)\left(y'(x_k)-\frac{h}{2}k''(x_k)+\bigO(h^2)\right)=
\end{multline*}
\[=hk'(x_k)y'(x_k)+hk(x_k)y''(x_k)+\bigO(h^3)=h(k(x_k)y'(x_k))'+\bigO(h^3)\]
Подставим получившееся значение и начальное условие в изначальное уравнение:
\[\abs{-(k(x_k)y'(x_k))'+\bigO(h^2)+p(x_k)y(x_k)-(-(k(x_k)y'(x_k))'+p(x_k)y(x_k))}\leq ch^2\]
Действительно, предложенная схема обладает вторым порядком аппроксимации на решении.

Построим для краевого условия задачи - краевого условия \textit{третьего рода} -
конечно-разностную аппроксимацию второго порядка точности на решении,
используя значения функции $y$ в точках $x_0=0$ и $x_1=h$.
Для простоты возьмем $k(x)\equiv1$. Воспользуемся $\delta$-поправкой.

Хотим получить следующее:
\[\abs{ay(0)+b\frac{y(h)-y(0)}{h}-c-\delta}\leq ch^2\]

Из формулы Тейлора в точке $0$ имеем:
\[\abs{ay(0)+b\frac{y(0)+hy'(0)+\frac{h^2}{2}y''(0)+\bigO(h^3)-y(0)}{h}-c-\delta}\leq ch^2\]
\[\abs{ay(0)+by'(0)+\frac{bh}{2}y''(0)+\bigO(h^2)-c-\delta}\leq ch^2\]
\[\abs{ay(0)+by'(0)+\frac{bh}{2}(p(0)y(0)-f(0))+\bigO(h^2)-c-\delta}\leq ch^2\]
Чтобы достигалось соответствующее неравенство
требуется взять $\delta:=\frac{bh}{2}(p(0)y(0)-f(0))$.

Таким образом аппроксимация второго порядка на краевом условии имеет вид
\[ay_0+b\frac{y_1-y_0}{h}=c+\frac{bh}{2}(p(0)y_0-f(0))\]

\subsection*{Примеры}
Рассмотрим разностные схемы для уравнения $y''(x)=f(x)$
\begin{example}
  Естественная аппроксимация:
  \[\frac{y_{i+1}-2y_i+y_{i-1}}{h^2}=f_k\]
  Главный член погрешности на решении равен
  \[r_h:=L_h(y)_h-f_h=\frac{h^2}{12}y^{(4)}(x_k)+\bigO(h^4)\]
\end{example}
\begin{example}
  Схема
  \[\frac{y_{i+1}-2y_i+y_{i-1}}{h^2}=f_k\frac{h^2}{12}f''(x_k)\]
  аппроксимирует уравнение на решении с порядком $\bigO(h^4)$
  (см. предыдущий пример и пользуемся тем, что $y''=f$).
\end{example}
\begin{example}
  Схема Нумерова
  \[\frac{y_{i+1}-2y_i+y_{i-1}}{h^2}=\frac{f_{k+1}+10f_k+f_{k-1}}{12}\]
  Главный член погрешности имеет вид
  \[r_h:=L_h(y)_h-f_h=\frac{h^2}{12}y^{(4)}(x_k)+\frac{h^4}{360}y^{(6)}(x_k)-\frac{h^2}{12}f^{(2)}(x_k)-\frac{h^4}{144}f^{(4)}(x_k)=-\frac{h^4}{240}y^{(6)}(x_k)+\bigO(h^6)\]
  Такую схему используют, если невозможно посчитать $f''$.
\end{example}
