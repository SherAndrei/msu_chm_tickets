\section{Построение многочленов Чебышёва первого и второго рода.}

Рассматривается рекурентное соотношение
\[y_{n+1}(x)=2xy_n(x)-y_{n-1}(x),\ x=\const\in\R\]
Так как $x$ зафиксировано, то можем решить
это соотношение как однородное разностное уравнение.
\[P(\mu)=\mu^2-2x\mu+1=0\Rightarrow\mu_{1,2}=x\pm\sqrt{x^2-1}\]
\begin{equation}\label{cheb::diff}
  \begin{array}{cc}
    |x|\neq1: & y_n(x)=C_1(x)(x+\sqrt{x^2-1})^n+C_2(x)(x-\sqrt{x^2-1})^n \\
    |x|=1:    & y_n(x)=C_1(x)+C_2(x)n
  \end{array}
\end{equation}
При $|x|<1$ сделаем замену $x=\cos\varphi$ и получим тригонометрическую форму записи:

\[y_n(x)=\hat{C}_1(x)(\cos(n\arccos(x)))+\hat{C}_2(x)(\sin(n\arccos(x)))\]

\begin{definition}
  Многочленами Чебышёва первого рода называется последовательность
  многочленов, удовлетворяющих рекурентному соотношению
  \[T_{n+1}(x)=2xT_n(x)-T_{n-1}(x),\ T_0(x)=1,\ T_1(x)=x\]
\end{definition}
\begin{theorem}
  \[T_n(x)=\frac{(x+\sqrt{x^2-1})^n+(x-\sqrt{x^2-1})^n}{2}\ \forall x\in\R\]
\end{theorem}
\begin{proof}
  Найдем $C_1$ и $C_2$ для первого уравнения системы \eqref{cheb::diff}
  \[\begin{cases}
      C_1(x+\sqrt{x^2-1})^0+C_2(x-\sqrt{x^2-1})^0=1 \\
      C_1(x+\sqrt{x^2-1})^1+C_2(x-\sqrt{x^2-1})^1=x \\
    \end{cases}\Rightarrow\begin{cases}
      C_1+C_2=1 \\
      (1-2C_2)\sqrt{x^2-1}=0
    \end{cases}\Rightarrow C_1=C_2=\frac{1}{2},\ x\neq\pm1\]
  Так как $T_n$ порождается полиномами - непрерывными функциями,
  а предложенная в теореме запись - рациональная функция - непрерывна,
  то значит что в точках $x\pm1$ из-за непрерывности они будут совпадать.
\end{proof}
\begin{theorem}
  \[T_n(x)=\cos(n\arccos(x)),\ |x|\leq1\]
\end{theorem}
\begin{proof}
  Найдем $C_1$ и $C_2$ для тригонометрической формы
  \[\begin{cases}
      C_1(\cos(0\cdot\arccos(x)))+C_2(\sin(0\cdot\arccos(x)))=1 \\
      C_1(\cos(\arccos(x)))+C_2(\sin(\arccos(x)))=x             \\
    \end{cases}\Rightarrow\begin{cases}
      C_1=1 \\
      C_2(\sin(\arccos(x)))=0
    \end{cases}\Rightarrow\begin{cases}
      C_1=1 \\
      C_2=0
    \end{cases}\ |x|\leq1\]
  По непрерывности аналогично получаем значения на краях интервала $(-1,1)$.
\end{proof}

\begin{definition}
  Многочленами Чебышёва второго рода называется последовательность
  многочленов, удовлетворяющих рекурентному соотношению
  \[U_{n+1}(x)=2xU_n(x)-U_{n-1}(x),\ U_0(x)=1,\ U_1(x)=2x\]
\end{definition}
\begin{theorem}
  \[U_n(x)=\frac{(x+\sqrt{x^2-1})^{n+1}-(x-\sqrt{x^2-1})^{n+1}}{2\sqrt{x^2-1}}\ \forall x\in\R\]
\end{theorem}
\begin{proof}
  Уже знаем, что $\forall\ x\in\R$ решение можно представить в виде
  \[U_n(x)=C_1(x)(x+\sqrt{x^2-1})^n+C_2(x)(x-\sqrt{x^2-1})^n\]
  Осталось подобрать коэффициенты $C_1$ и $C_2$:
  \begin{multline*}
    \begin{cases}
      C_1(x+\sqrt{x^2-1})^0+C_2(x-\sqrt{x^2-1})^0=1  \\
      C_1(x+\sqrt{x^2-1})^1+C_2(x-\sqrt{x^2-1})^1=2x \\
    \end{cases}\Rightarrow\begin{cases}
      C_1+C_2=1 \\
      (1-2C_2)\sqrt{x^2-1}=x
    \end{cases}\Rightarrow \\ \Rightarrow\begin{cases}
      C_1+C_2=1 \\
      C_2=\frac{\sqrt{x^2-1} - x}{2\sqrt{x^2-1}}
    \end{cases}\Rightarrow \begin{cases}
      C_1=\frac{\sqrt{x^2-1} + x}{2\sqrt{x^2-1}} \\
      C_2=-\frac{x-\sqrt{x^2-1}}{2\sqrt{x^2-1}}
    \end{cases}\ x\neq\pm1
  \end{multline*}
  Аналогично из непрерывности следует тождественность на $x=\pm1$
\end{proof}

\begin{theorem}
  \[U_n(x)=\frac{\sin((n+1)\arccos{x})}{\sin\arccos{x}},\ |x|\leq1\]
\end{theorem}
\begin{proof}
  Сделаем замену $x=\cos\varphi$ в предыдущей теореме и воспользуемся формулой Муавра
  \begin{multline*}
    \frac{(\cos\varphi+i\sin\varphi)^{n+1}-(\cos\varphi-i\sin\varphi)^{n+1}}{2i\sin\varphi}= \\
    \frac{\cos((n+1)\varphi)+i\sin((n+1)\varphi)-\cos((n+1)\varphi)+i\sin((n+1)\varphi)}{2i\sin\varphi}= \frac{2i\sin((n+1)\varphi)}{2i\sin\varphi}=\frac{\sin((n+1)\arccos{x})}{\sin\arccos{x}}
  \end{multline*}
  Аналогично из непрерывности получаем тождественность для $x=\pm1$
\end{proof}
\begin{remark*}
  Обратим внимание, что $T_n(x)$ и $U_n(x)$ порождаются линейно независимыми
  комбинациями $(1,x)$ и $(1,2x)$. Это значит, что любое рекурентное
  соотношение $y_{n+1}(x)=2xy_n(x)-y_{n-1}(x)$ имеет решение
  $y_{n}(x)=C_1(x)T_n(x)+C_2(x)U_n(x)$.
\end{remark*}
