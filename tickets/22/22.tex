\section{Метод стрельбы и метод Фурье. Численные методы линейной алгебры.}

\subsection*{Метод стрельбы}

Решаем следующую конечно-разностную схему
\[-\frac{y_{k+1}-2y_k+y_{k-1}}{h^2}=f,\ y_0=a,\ y_N=b\Leftrightarrow L_hy_h=f_h\]
Если бы у нас было начальное условие $y_1=c$, то мы могли бы
однозначно построить решение, используя методы решения разностных
уравнений, но у нас такого нет. Идея этого метода следующая:
рассмотрим две вспомогательные задачи
\[L_hu_h=f_h,\ u_0=a,\ u_1=\hat{u}\]
\[L_hw_h=0,\ w_0=0,\ u_1=\hat{w}\]
Такие две задачи мы можем решить рекуррентно пересчитывая каждое
следующее значение. Итоговое решение предлагается искать в виде
\begin{equation}\label{eq:shooting_solution}
  y_h=u_h+Cw_h
\end{equation}
где $C$ мы получим из второго начального
условия изначальной задачи. Почему это верно? Применим
разностный оператор к обеим частям:
\[L_hy_h = L_hu_h + C\underbrace{L_hw_h}_{=0}=f_h\]
Ищем $C$ следующим образом: рекуррентно посчитаем $u_N$ и $w_N$
и подставим в \eqref{eq:shooting_solution}: $C=\frac{y_N-u_N}{w_N}=\frac{b-u_N}{w_N}$

Формально значения $\hat{u}$ и $\hat{w}$ могут быть произвольными,
но разумно выбирать их так, чтобы $u_1-u_0=O(h)$, $w_1-w_0=O(h)$.
Потому что переход от $u_0$ к $u_1$ идет за один шаг ($h$),
а если выбрать очень их большими, то итоговый ответ будет испорчен
из-за погрешности вычислений.

\begin{remark}
  Название метода берет свое начало из аналогичного метода
  прицеливания артиллерией: сначала наводятся на цель, используя
  одни начальные данные, и смотрят на результирующую траекторию снаряда,
  затем выбирают другие начальные данные, получают другую траекторию,
  и в конце высчитывают те начальные данные, которые точно гарантируют попадание.
\end{remark}

\subsection*{Метод Фурье}

Требуется найти решение СЛАУ $Ay=f,\ y,f\in\R,\ A\in\R^{M\times M}$.
Известны собственные вектора и собственные числа матрицы $Ay^{(n)}=\lambda_ny^{(n)},\ n=1,\ldots,M$
и $y^{(n)}$ образуют ортонормированный базис в $R^M$.

Идея метода состоит в разложении решения $y$ по базисным векторам
$y=\sum_{n=1}^Mc_ny^{(n)}$, определении коэффициентов $c_n$
и последующем восстановлении $y$.

\begin{remark}
  Вспомним, что нахождение определителя матрицы является неустойчивой
  задачей, и, соответственно, обычный способ нахождения собственных
  значений через характеристический многочлен нам не подходит.
  Существуют и другие способы их нахождения, но сложность
  таких алгоритмов не меньше использования алгоритма с нахождением
  определителя. Поэтому метод Фурье чаще используют тогда,
  когда все собственные числа и вектора возможно найти аналитически,
  и система собственных векторов образует ортонормированный базис в
  пространстве решений относительно зафиксированного
  скалярного произведения $(y^{(m)},y^{(n)})_h=\delta_m^n$.
  Когда все эти условия соблюдаются, то коэфффициенты $c_n$
  могут быть найдены по явной формуле.
\end{remark}

Перепишем исходную задачу, выразив решение через базис
\[A\left(\sum_{n=1}^Mc_ny^{(n)}\right)=b\Leftrightarrow\sum_{n=1}^M\lambda_nc_ny^{(m)}=b\]
Пройдемся по всем $m=1,\ldots,M$, скалярное умножим данное равенство на $y^{(m)}$
и воспользуемся ортонормированностью базиса
\[\left(\sum_{n=1}^M\lambda_nc_ny^{(n)}, y^{(m)}\right)_h=(b,y^{(m)})_h\Leftrightarrow\lambda_mc_m=(b,y^{(m)})_h\]
Рассмотрим правую часть: аналогично разложим $b$ по базису и получим $\left(\sum_{m=1}^Md_my^{(m)}y,y^{(m)}\right)_h=d_m$.
Отсюда имеем $c_m=\frac{d_m}{\lambda_m}$. Итоговый ответ имеет вид
\[y=\sum_{m=1}^M\frac{d_m}{\lambda_m}y^{(m)}\]

\begin{remark}
  Обратим внимание, что в получившемся решении $\lambda_m\neq0$.
  Иначе если $\lambda_m=0$, то $\det{A}=0$, а это не может
  нам гарантировать корректность исходной задачи, то есть решение
  исходной задачи не является единственным.
\end{remark}

\begin{remark}
  По сравнению с методом прогонки, у которого сложность $O(N)$,
  у метода Фурье в общем случае $\bigO(N^2)$.
  Эту сложность можно уменьшить за счет так называемого быстрого
  преобразования Фурье ($\bigO(N\log{N})$).
\end{remark}

\begin{example}
  Решим задачу $-y''+py=f,\ p\equiv\const\geq0,\ y(0)=y(1)=0$ методом Фурье.
  Воспользуемся следующей конечно-разностной схемой
  \[-\frac{y_{k+1}-2y_{k}+y_{k-1}}{h^2}+py_k=f_k,\ y_0=y_N=0,\ k=1,\ldots,N-1\]

  Выберем согласованное скалярное произведение: $(u,v)_h=h\sum_{i=1}^{N-1}u_iv_i$.

  Матричная запись данной разностной схемы имеет вид
  \[\left[-\left(\begin{array}{cccccc}
        \frac{-2}{h^2} & \frac{1}{h^2}  &               &               &                & 0              \\
        \frac{1}{h^2}  & \frac{-2}{h^2} & \frac{1}{h^2} &               &                &                \\
                       &                & \cdots        & \cdots        &                &                \\
                       &                &               & \frac{1}{h^2} & \frac{-2}{h^2} & \frac{1}{h^2}  \\
        0              &                &               &               & \frac{1}{h^2}  & \frac{-1}{h^2} \\
      \end{array}\right)
      +
      \left(\begin{array}{ccc}
        p_1 &        & 0       \\
            &        &         \\
            & \ddots &         \\
            &        &         \\
        0   &        & p_{N-1} \\
      \end{array}\right)\right]
    \left(\begin{array}{c}
        y_{1}   \\
        \\
        \vdots  \\
        \\
        y_{N-1} \\
      \end{array}\right)
    =\left(\begin{array}{c}
        f_{1}   \\
        \\
        \vdots  \\
        \\
        f_{N-1} \\
      \end{array}\right)
    \Leftrightarrow\left[\tilde{A}+pI\right]y=f\]

  Решение данной разностной задачи может быть найдено аналитически
  \[y_{k}^{(m)}=C\sin(\pi kmh),\ \lambda_m = \frac{4}{h^2}\sin^2\left(\frac{\pi m h}{2}\right)+p,\ k=0,\ldots,N,\ m=1,\ldots,N-1\]

  \begin{remark}
    Обратим внимание почему $p\equiv\const\geq0$
    \[\tilde{A}\tilde{e_i}=\tilde{\lambda_i}\tilde{e_i}\Rightarrow(A-pI)\tilde{e_i}=A\tilde{e_i}-p\tilde{e_i}=\tilde{\lambda_i}\tilde{e_i}\Leftrightarrow A\tilde{e_i}=(\tilde{\lambda_i}+p)\tilde{e_i}\]
    Мы можем ослабить условие на $p$, сохраняя необходимое условие метода Фурье:
    $\lambda_m=\frac{4}{h^2}\sin^2\left(\frac{\pi m h}{2}\right)+p\neq0$
  \end{remark}

  Проверим, что для полученного решения выполняются необходимые условия
  для применения метода Фурье: ортогональность векторов $y^{(m)}$
  гарантируется так как исходная матрица является симметричной.
  Подберем $C$ так, чтобы выполнялась ортонормированность:
  \[(y^{(n)},y^{(n)})_h=h\sum_{k=1}^{N-1}(y^{(n)}_k)^2=hC^2\sum_{k=1}^{N-1}\sin^2(\pi knh)=\frac{hC^2}{2}\sum_{k=1}^{N-1}(1-\cos(2\pi knh))=\frac{hC^2}{2}(N-1)-\frac{hC^2}{2}\cdot\star\]
  \[\star: \sum_{k=1}^{N-1}\cos(2\pi knh)=\text{Re}\sum_{k=1}^{N-1}e^{2\pi iknh}=\text{Re}\left(\underbrace{\sum_{k=1}^{N}e^{2\pi iknh}}_{=0}-e^{2\pi inN\frac{1}{N}}\right)=-\text{Re}(e^{2\pi in})=-1\]
  \[(y^{(n)},y^{(n)})_h=\frac{hC^2}{2}(N-1)+\frac{hC^2}{2}=1\Leftrightarrow\frac{NhC^2}{2}=1\Leftrightarrow C=\sqrt{2}\]
  Таким образом, при $C=\sqrt{2}$ гарантируем ортонормированность собственных векторов. Можем применить метод Фурье.
\end{example}

\begin{example}
  Решим задачу $-y''=f,\ y'(0)=y'(1)=0$ методом Фурье.
  Воспользуемся следующей конечно-разностной схемой
  \[\begin{cases}
      -\frac{y_{k+1}-2y_{k}+y_{k-1}}{h^2}=f_k,\ k=1,\ldots,N-1,\ h=1/N \\
      \frac{2}{h^2}(y_1-y_0)=f_0                                       \\
      \frac{2}{h^2}(y_{N}-y_{N-1})=f_N
    \end{cases}
  \]

  Выберем согласованное скалярное произведение: $(u,v)_h=h\sum_{i=1}^{N-1}u_iv_i+\frac{h}{2}u_0v_0+\frac{h}{2}u_Nv_N$.

  Матричная запись данной разностной схемы имеет вид
  \[-\left(\begin{array}{cccccc}
        \frac{-2}{h^2} & \frac{2}{h^2}  &               &               &                & 0              \\
        \frac{1}{h^2}  & \frac{-2}{h^2} & \frac{1}{h^2} &               &                &                \\
                       &                & \cdots        & \cdots        &                &                \\
                       &                &               & \frac{1}{h^2} & \frac{-2}{h^2} & \frac{1}{h^2}  \\
        0              &                &               &               & \frac{2}{h^2}  & \frac{-2}{h^2} \\
      \end{array}\right)
    \left(\begin{array}{c}
        y_{0}  \\
        \\
        \vdots \\
        \\
        y_{N}  \\
      \end{array}\right)
    =\left(\begin{array}{c}
        f_{0}  \\
        \\
        \vdots \\
        \\
        f_{N}  \\
      \end{array}\right)\]

  Решение данной разностной задачи может быть найдено аналитически
  \[y_{k}^{(m)}=C\cos(\pi kmh),\ \lambda_m = \frac{4}{h^2}\sin^2\left(\frac{\pi m h}{2}\right),\ k=0,\ldots,N,\ m=0,\ldots,N\]

  Обратим внимание, что матрица не является симметричной в метрике
  исходной задачи. Покажем, что в выбранной метрике матрица как
  оператор, действующий на вектор $y$, является симметричной:
  \[(Au,v)_h=h\sum_{i=1}^{N-1}\sum_{j=0}^{N}a_{ij}u_iv_i+\frac{h}{2}\sum_{j=0}^{N}a_{0j}u_0v_0+\frac{h}{2}\sum_{j=0}^{N}a_{Nj}u_Nv_N=h(\tilde{A}u,v),\ \frac{1}{2}a_{0j}=\tilde{a}_{0j},\ \frac{1}{2}a_{Nj}=\tilde{a}_{Nj}\]
  Но матрица $\tilde{A}=\tilde{A}^T$, следовательно:
  \[(Au,v)_h=h(\tilde{A}u,v)=h(u,\tilde{A}v)=h(\tilde{A}v,u)=(Av,u)_h=(u,Av)_h\]

  Проделав шаги, аналогичные предыдущему примеру, для соблюдения ортонормированности
  получим соответствующие константы $C_k=\sqrt{2},\ k=1,\ldots,N-1$, $C_0=C_N=1$.

  Отметим, что $\lambda_0=0$, $y^{(0)}\equiv1$. Но для применимости метода Фурье
  требуется, чтобы собственные значения были не нулевые. Вспомним
  для чего это требуется: при поиске коэффициентов $c_m$
  в разложении решения в базисе собственных векторов мы пришли к
  равенству $\lambda_mc_m=(b,y^{(m)})_h$. Тогда необходимым
  и достаточным условием корректности алгоритма требуется, чтобы
  $(b,1)_h=0$.
\end{example}
