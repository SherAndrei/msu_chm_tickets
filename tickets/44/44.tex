\section{Многочлен наилучшего равномерного приближения. Теорема Валле–Пуссена. Теорема Чебышёва.}
% Best Uniform Approximation Polynomial
Пусть $f:\sup_{x\in[a,b]}|f(x)|<\infty$ - функционал огр. вариации, $\{g^{(i)}\}_{i=1}^{n+1}=\{1,x,\ldots,x^n\}$. Решаем задачу
\begin{equation}\label{contig::task}
  \inf_{a_0,\ldots,a_n}\max_{x\in[a,b]}\left|f(x)-\sum_{i=1}^na_ix^i\right|=\inf_{Q_n}\Vert f-Q_n\Vert_{C[a,b]}
\end{equation}

$\exists$ решение (т.к. пространство полн. лин.), проверим единственность.
\begin{definition}
  $Q_n^o(x)$ - многочлен наилучшего равномерно приближения (МНРП) для $f(x)$ на $[a,b]$,
  если $$\forall\ Q_n\ \Vert f- Q_n\Vert_{C[a,b]}\geq\Vert f- Q_n^o\Vert_{C[a,b]}$$
\end{definition}
\begin{remark*}
  $Q_n^o$ - в общем случае \underline{не} многочлен Чебышева, т.к. он приближает 0,
  но МНРП для 0 есть 0. Разница в том, что в мн-не Чебышева зафиксирован свободный член, тогда
  как здесь участвуют \underline{все} коэф.
\end{remark*}
\begin{remark*}
  Такой многочлен существует всегда (по теореме об элементе наилучшего
  приближения в линейном нормированном пространстве), а его единственность (см. далее) имеет место для непрерывных функц
\end{remark*}
\begin{theorem}[Валле-Пуссен]
  $f(x)$, $Q_n(x)$, $\exists\ n+2$ точки, $a\leq x_0\leq\ldots\leq x_{n+1}\leq b$
  $$\sgn{\left(f(x_i)-Q_n(x_i)\right)}\cdot (-1)^i\equiv\const$$
  т.е. при переходе от точки к точке разность $f(x_i)-Q_n(x_i)$ меняет знак. Тогда
  $$\Vert f- Q_n^o\Vert\geq\mu=\min_{i=0,\ldots,n+1}\left|f(x_i)-Q_n(x_i)\right|$$
\end{theorem}
\begin{theorem}[Чебышев]
  Пусть $f\in C[a,b]$ тогда $Q_n^o$ - МНРП $\Leftrightarrow$ на отрезке $[a,b]$ $\exists$
  по крайней мере $n+2$ точек $a\leq x_0\leq\ldots\leq x_{n+1}\leq b$:
  $$f(x_i)-Q_n^o(x_i)=\alpha(-1)^i\Vert f-Q_n^o\Vert$$
  где $i=0,\ldots, n+1;$ $\alpha=1$ или $\alpha=-1$ одновременно для всех $i$.
\end{theorem}
Точки $x_0,\ldots,x_{n+1}$, удовлетворяющие условию теоремы, называются точками Чебышёвского альтернанса.
