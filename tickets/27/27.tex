\section{Задача наименьших квадратов неполного ранга: методы QR-разложения и QR-разложения с выбором главного столбца}

\subsection*{Методы QR-разложения}

ЗНК называется вырожденной, если $\text{rk}(A)<n$, $\det(A^TA)=0$. При численном решении
вырожденных и почти вырожденных систем требуется изменить постановку задачи
и соответственно применять иные методы.

\begin{remark}
  Рассмотрим следующий пример
  \[\left(\begin{array}{cc}
        1 & 1 \\
        0 & 0
      \end{array}\right)\left(\begin{array}{c}
        x_1 \\
        x_2
      \end{array}\right)=\left(\begin{array}{c}
        1 \\
        1
      \end{array}\right)\]

  В данном уравнении $m=n=2$ и задача вырождена, можно выбрать семейство решений
  $\mathbf{x}=(1-x_2,x_2)^T,\ \mathbf{r}=\mathbf{b}-A\mathbf{x}=(0,1)^T$.
  Можно выбрать решения как с нормами порядка единицы, так и со сколь угодно большими.
  Однако, для возмущенной задачи

  \[\left(\begin{array}{cc}
        1 & 1           \\
        0 & \varepsilon
      \end{array}\right)\left(\begin{array}{c}
        x_1 \\
        x_2
      \end{array}\right)=\left(\begin{array}{c}
        1 \\
        1
      \end{array}\right)\]

  формально близкой к исходной при $\varepsilon\approx 0$, имеется
  единственное решение $(1-\varepsilon^{-1},\varepsilon^{-1})^T$ с большой
  нормой порядка $\varepsilon^{-1}$ и нулевой невязкой. Это
  означает, что сколь угодно малое возмущение элементов матрицы может
  существенно изменить структуру и норму решения.
\end{remark}

\begin{theorem}
  Пусть дана $A\in\R^{m\times n}$, $\text{rk}(A)=k<n\leq m$.
  Тогда множество векторов - решений ЗНК - образует
  линейное подпространство размерности $n-k$.
\end{theorem}
\begin{proof}
  Так как $\text{rk}(A)=k<n$, то $\dim(\ker(A))=n-k$.
  Рассмотрим $\mathbf{y}\in\ker(A)$. Утвержается, если $\mathbf{x}$ - решение ЗНК,
  то есть $\inf_{\mathbf{z}}\norm{\mathbf{b}-A\mathbf{z}}_2=\norm{\mathbf{b}-A\mathbf{x}}$,
  то $\mathbf{x}+\mathbf{y}$ является решением ЗНК. Действительно
  \[\norm{\mathbf{b}-A(\mathbf{x}+\mathbf{y})}_2=\norm{\mathbf{b}-A\mathbf{x}-A\mathbf{y}}_2=\norm{\mathbf{b}-A\mathbf{x}}_2\]
\end{proof}

\begin{theorem}
  Пусть дана $A\in\R^{m\times n}$, $\text{rk}(A)=k<n\leq m$.
  Тогда матрицу можно привести к следующему виду
  \[A=QR=\left(\begin{array}{cc|c}
        Q_1       &  & Q_2           \\
        m\times k &  & m\times (m-k)
      \end{array}\right)
    \left(\begin{array}{c|c}
        R_1              & R_2                  \\
        k\times k        & k\times (n-k)        \\
        \hline
        0                & 0                    \\
        (m - k) \times k & (m - k) \times (n-k)
      \end{array}\right)\]
  $Q\in\R^{m\times m}$, $Q^TQ=I$, $R_1$ - верхнетреугольная, $\det(R_1)\neq0$.
  Для ЗНК с матрицей $A$ имеется семейство решений
  \[\mathbf{x}=(R_{11}^{-1}(Q_1\mathbf{b}-R_{12}\mathbf{x_2}), \mathbf{x}_2)^T,\ \mathbf{x}_1\in\R^{k},\ \mathbf{x}_2\in\R^{n-k}\]
\end{theorem}
\begin{proof}
  Для того, чтобы понять почему матрицы имеют такой вид,
  построим матрицу $Q$ размера $m\times m$ как ортонормальные вектора из $\R^m$
  с помощью алгоритма Грама-Шмидта. Посмотрим на произведение следующих матриц
  \[Q^TA=\left(\begin{array}{c}
        \mathbf{q}_1     \\
        \vdots           \\
        \mathbf{q}_k     \\
        \mathbf{q}_{k+1} \\
        \vdots           \\
        \mathbf{q}_m
      \end{array}\right)\Bigl(\mathbf{a}_1,\ldots,\mathbf{a}_k,\mathbf{a}_{k+1},\ldots,\mathbf{a}_n\Bigr)
    =\left(\begin{array}{c|c}
        R_{11}        & R_{12}              \\
        k \times k    & k \times (n - k)    \\
        \hline
        R_{21}        & R_{22}              \\
        (m-k)\times k & (m-k)\times (n - k)
      \end{array}\right)
  \]

  \begin{enumerate}
    \item $R_{11}=\left(\begin{array}{ccc}
                (\mathbf{q}_1,\mathbf{a}_1) &        & (\mathbf{q}_1,\mathbf{a}_k) \\
                                            & \ddots &                             \\
                (\mathbf{q}_k,\mathbf{a}_1) &        & (\mathbf{q}_k,\mathbf{a}_k)
              \end{array}\right) = \left(\begin{array}{ccc}
                (\mathbf{q}_1,\mathbf{a}_1) &        & (\mathbf{q}_1,\mathbf{a}_k) \\
                                            & \ddots &                             \\
                0                           &        & (\mathbf{q}_k,\mathbf{a}_k)
              \end{array}\right)$ и $R_{21}=0$ так как $(\mathbf{q}_i,\mathbf{a}_j)=0,\ i > j$
    \item Рассмотрим произвольный элемент матрицы $R_{12}$: $(\mathbf{q}_i,\mathbf{a}_j)$,
          $1 \leq i \leq k$, $k < j \leq n$. Так как $\text{rk}(A)=k$, то $\mathbf{a}_j$
          можно выразить через $\mathbf{q}_t,\ 1 \leq t \leq k$: $\left(\mathbf{q}_i,\sum_{t=1}^{k}c_{t}\mathbf{q}_t\right)=c_i$.
          То есть матрица $R_{12}$ состоит из компонент векторов разложенных по $\mathbf{q}_i$, $1 \leq i \leq k$.
          И так как в разложении этих векторов не участвуют $\mathbf{q}_j$, $k < j \leq n$,
          то соответствующие коэффициенты в разложении равны $0$, а значит $R_{22}=0$.
  \end{enumerate}
  Решим задачу наименьших квадратов. Исходную задачу домножим на $Q^T$. Итоговая норма
  не изменится, так как $Q^T$ является ортогональной матрицей, то есть сохраняет длину.
  \[\inf_{\tilde{\mathbf{x}}}\norm{\mathbf{b}-A\tilde{\mathbf{x}}}_2^2
    =\inf_{\tilde{\mathbf{x}}}\norm{Q^T\mathbf{b}-R\tilde{\mathbf{x}}}_2^2
    =\inf_{\tilde{\mathbf{x}}}\norm{\left(\begin{array}{c} Q_1^T \\ Q_2^T \end{array}\right)\mathbf{b}-\left(\begin{array}{cc} R_{11} & R_{12} \\ 0 & 0 \end{array}\right)\left(\begin{array}{c}\tilde{\mathbf{x}}_1 \\ \tilde{\mathbf{x}}_2\end{array}\right)}_2^2\]
  \[=\inf_{\tilde{\mathbf{x}}}\norm{\begin{array}{c} Q_1^T\mathbf{b}- R_{11}\tilde{\mathbf{x}}_1-R_{12}\tilde{\mathbf{x}}_2 \\ Q_2^T\mathbf{b} \end{array}}_2^2
    =\inf_{\tilde{\mathbf{x}}}\left(\norm{Q_1^T\mathbf{b}- R_{11}\tilde{\mathbf{x}}_1-R_{12}\tilde{\mathbf{x}}_2}_2^2 + \norm{Q_2^T\mathbf{b}}^2_2\right)\]
  Минимум будет достигаться только если нам удастся занулить первое слагаемое.
  Тогда решение можно молучить из равенства
  \[\norm{Q_1^T\mathbf{b}- R_{11}\mathbf{x}_1-R_{12}\mathbf{x}_2}_2^2=0\Leftrightarrow \mathbf{x}_1=R^{-1}_{11}(Q_1^T\mathbf{b}+R_{12}\mathbf{x}_2),\ \forall\ \mathbf{x}_2\in\R^{n-k}\]
  Таким образом, пространство решений имеет размер $n-k$.
\end{proof}
\begin{remark}
  Научились решать задачу $\inf_{\tilde{\mathbf{x}}}\norm{\mathbf{b}-A\tilde{\mathbf{x}}}_2$,
  но что если от нас требуется найти $\tilde{\mathbf{x}}_2$: $\norm{\mathbf{x}}\rightarrow\inf$?
  Обычно выбор $\tilde{\mathbf{x}_2}\equiv0$ может давать неплохое приближение, но
  в общем случае бывают задачи, где такой ответ не подходит.
\end{remark}

\subsection*{$QR$-разложение с выбором главного столбца}

Преобразуем исходную задачу так, чтобы первые $k$
столбцов полученной матрицы $A$ были линейно независимы.
Перестановки столбцов в матрице $A$ удобно проводить
в процессе вычислений. Цель соответствующих перестановок - получить в
матрице $R=\left(\begin{array}{cc}
      R_{11} & R_{12} \\
      0      & R_{22} \\
      0      & 0
    \end{array}\right)$ как можно лучше обусловленный блок $R_{11}$ и
как можно меньшие элементы в $R_{22}$. В машинной арифметике
принято считать, что блок $R_{22}$ обнулился, если все вектора в нем
размера меньше фиксированного машинного $\varepsilon$.

Соответствующие вычисления проводятся на основе стандартного $QR$-
разложения, и так как удобнее проводить алгоритм итеративно,
то чаще используется метод отражений.
На $k$-м шаге, $1\leq k \leq n-1$, в матрице $A$ выбирают столбец
с номером $j_k$, $k\leq j_k\leq n$, с наибольшей величиной $\max_{k\leq j\leq n}\norm{\mathbf{a}^{(k)}_{j}}_2$:
Если таких столбцов несколько, то берут произвольный из них. В матрице
$A$ найденный столбец $j$ переставляют с $k$-м столбцом.
Далее реализуют очередной шаг $QR$-разложения.

В конце алгоритма переставляют компоненты решения $x_i$ в соотвествии
с соответствии с перестановками столбцов.

\begin{remark}
  Для невырожденной задачи $A\mathbf{x}=\mathbf{b}$ с матрицей $A\in\R^{n\times n}$ вида
  \[A=\left(\begin{array}{ccccc}
        1 & -1 & \ldots &   & -1     \\
        0 & 1  &        &   &        \\
          &    & \ddots &   & \vdots \\
          &    &        & 1 & -1     \\
        0 &    &        &   & 1      \\
      \end{array}\right)\]
  $QR$-алгоритм с выбором максимального элемента даст результат
  намного хуже, чем стандартный. При стандартном $QR$ ответ
  записывается сразу: $Q\equiv I$, $R\equiv A$.
  Тогда как при выборе максимального последний элемент $R_{nn}$
  будет иметь порядок $\bigO(2^{-n})$. То есть, если мы возьмем $n=50$,
  то последний элемент будет сравним с машинным нулем, и алгоритм
  вполне может посчитать получившуюся матрицу вырожденной,
  хотя задача очевидно таковой не является.
\end{remark}
