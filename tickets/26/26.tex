\section{Невырожденная ЗНК: метод нормального уравнения, метод QR-разложения.}

Пусть требуется решить СЛАУ $A\mathbf{x}=\mathbf{b}$ с матрицей $A$
размерности $m\times n$, правой частью $\mathbf{b}\in\R^m$ и вектором
неизвестных $x\in\R^n$. Рассмотрим три случая:
\begin{enumerate}
  \item $m = n$, $\det(A)\neq0$.
        Задача невырождена и вектор $\mathbf{x}=A^{-1}\mathbf{b}$ является точным решением.
        Для вектора невязки $\mathbf{r}=\mathbf{b}-A\mathbf{x}$ имеем $\norm{\mathbf{r}}=0$.
        \[n\underset{n}{\left(\begin{array}{ccc}
               &   & \\
               & A & \\
               &   & \\
            \end{array}\right)}\underset{n}{\left(\begin{array}{c}
              \\ \mathbf{x} \\ \\
            \end{array}\right)}=\underset{n}{\left(\begin{array}{c}
              \\ \mathbf{b} \\ \\
            \end{array}\right)}\]
  \item $m < n$, $\text{rk}(A)=m$.
        Задача недоопределена. Исходная система имеет подпространство решений размерности $n-m$.
        \[m\underset{n}{\left(\begin{array}{ccc}
               & A & \\
               &   & \\
            \end{array}\right)}\underset{n}{\left(\begin{array}{c}
              \\ \mathbf{x} \\ \\
            \end{array}\right)}=\underset{m}{\left(\begin{array}{c}
              \mathbf{b} \\  \\
            \end{array}\right)}\]
  \item $m > n$, $\text{rk}(A)=n$.
        Система переопределена и, если она несовместна, то точного решения не
        существует, т.е. для произвольного $\mathbf{x}\in\R^n$ имеем $\norm{\mathbf{b}-A\mathbf{x}}=\norm{r}>0$.
        \[m\underset{n}{\left(\begin{array}{ccc}
               &   & \\
               & A & \\
               &   & \\
               &   & \\
            \end{array}\right)}\underset{n}{\left(\begin{array}{c}
              \\ \mathbf{x} \\ \\
            \end{array}\right)}=\underset{m}{\left(\begin{array}{c}
              \\ \mathbf{b} \\ \\ \\
            \end{array}\right)}\]
\end{enumerate}

Представляют интерес методы решения переопределенных задач,
поэтому далее, если не оговаривается иное, считаем, что $m > n$, $\text{rk}(A)=n$.
Для задач такого рода Гаусс предложил считать решением вектор $\mathbf{x}$,
минимизирующий евклидову норму вектора невязки $\inf{\mathbf{y}}\norm{\mathbf{b}-A\mathbf{y}}_2=\norm{\mathbf{b}-A\mathbf{x}}_2$.

Рассмотрим некоторые методы решения данной минимизационной задачи,
называемой задачей наименьших квадратов (ЗНК).

\subsection*{Метод нормального уравнения}
Рассмотрим следующую, называемую нормальной, систему уравнений
$A^TA\mathbf{x}=A^T\mathbf{b}$ с квадратной матрицей $A^TA\in\R^{n\times n}$.
Отсюда найдем вектор $\mathbf{x}$.

\begin{theorem}[Гаусс К.Ф.]
  Пусть $m\geq n$, $\text{rk}(A)=n$. Тогда нормальное уравнение имеет единственное решение.
\end{theorem}
\begin{proof}
  Рассмотрим $B=A^TA$, тогда $B=B^T$, так как $B^T=(A^TA)^T=A^TA=B$ и
  \[(B\mathbf{x}, \mathbf{x})=(A^TA\mathbf{x}, \mathbf{x})=(A\mathbf{x}, A\mathbf{x})>0,\ \forall \mathbf{x}\neq0\]
  Значит, $B$ является положительно определенной, значит невырожденной, значит $\exists!$
  решение задачи.
\end{proof}
Покажем, что решение нормального уравнения является решением исходной задачи
\begin{theorem}
  Пусть $m\geq n$, $\text{rk}(A)=n$. Вектор $\mathbf{x}$ является решением
  ЗНК $\inf_\mathbf{y}\norm{\mathbf{b}-A\mathbf{y}}_{2}$ тогда
  и только тогда, когда $\mathbf{x}$ является решением нормального уравнения $A^TA\mathbf{x}=A^T\mathbf{b}$.
\end{theorem}
\begin{proof}
  Из предыдущей теоремы известно, что $\exists!\ \mathbf{x}$ - решение нормального уравнения, тогда
  \[A^TA\mathbf{x}-A^T\mathbf{b}=0\Leftrightarrow A^T(A\mathbf{x}-\mathbf{b})=0\]
  Возьмем $\mathbf{y}=\mathbf{x}+\mathbf{\Delta}$, тогда
  \[\norm{A\mathbf{y}-\mathbf{b}}^2_2=(A\mathbf{x}-\mathbf{b}+A\mathbf{\Delta},A\mathbf{x}-\mathbf{b}+A\mathbf{\Delta})_2=\norm{A\mathbf{x}-\mathbf{b}}_2^2+2(\Delta,\underbrace{A^T(A\mathbf{x}-\mathbf{b})}_{=0})_2+\norm{\mathbf{\Delta}}_2^2\]
  Следовательно, минимум достигается при $\mathbf{\Delta}=0$, то есть на векторе $\mathbf{y}=\mathbf{x}$.
\end{proof}

Метод нормального уравнения прост в реализации, однако в приближенной
арифметике для почти вырожденных задач большой размерности может давать
плохой результат. Например, в случае квадратной матрицы $A^T=A$ имеем
$\cnd_2(A^TA)=\cnd_2(A)^2$. Таким образом, обусловленность исходной задачи
возводится в квадрат, поэтому полученное численно решение может сильно отличаться от точного,
если $\cnd(A)\gg1$. Рассмотрим более устойчивый метод.

\subsection*{$QR$-разложение}

\begin{theorem}
  Пусть $A$ - $m\times n$ матрица, $m\geq n$, $\text{rk}(A)=n$.
  Тогда $\exists!$ матрица $Q\in\R^{m\times n}$, $Q^TQ=I_n$, и $R\in\R^{n\times n}$, $R_{ii}>0$ -
  верхнетреугольная такие, что $A=QR$.
\end{theorem}
\begin{proof}
  С помощью алгоритма ортогонализации Грама-Шмидта по набору $<\mathbf{a}_1,\ldots,\mathbf{a}_n>$
  из столбцов матрицы $A$ построим ортонормальный базис $<\mathbf{q}_1,\ldots,\mathbf{q}_n>$.
  По найденным векторам построим матрицу $Q=(\mathbf{q}_1,\ldots,\mathbf{q}_n)$ и рассмотрим $R=Q^TA$.
  \[Q^TA=
    \left(\begin{array}{c}
        \mathbf{q}_1^T \\
        \vdots         \\
        \mathbf{q}_n^T
      \end{array}\right)
    \left(\mathbf{a}_1,\ldots,\mathbf{a}_n\right)=
    \left(\begin{array}{cccc}
        (\mathbf{q}_1,\mathbf{a}_1) & (\mathbf{q}_1,\mathbf{a}_2) &        & (\mathbf{q}_1,\mathbf{q}_n) \\
        0                           & (\mathbf{q}_2,\mathbf{a}_2) &        & (\mathbf{q}_2,\mathbf{q}_n) \\
        \ldots                      & \ldots                      & \ddots & \ldots                      \\
        0                           & 0                           &        & (\mathbf{q}_n,\mathbf{q}_n) \\
      \end{array}\right)=R\]
  Обратим внимание, что $(\mathbf{q}_i,\mathbf{a}_j)=0$ при $i>j$ по построению
  базиса через алгоритм Грама-Шмидта. Так же построенная матрица $Q$
  является ортогональной, так как $(\mathbf{q}_i,\mathbf{q}_j)=\delta_{ij}$.
  Отсюда и из $Q^TA=R$ имеем $A=QR$.

  Единственность в данном случае следует из факта, что построить
  соответствующее линейное подпространство натянутое на $\{\mathbf{q}_i\}_i$
  можно единственным образом, если договориться, что элементы
  на диагонали матрицы $R$ положительны. Тогда каждый следующий вектор
  из базиса будет иметь направление совпадающее с направлением,
  полученным по правилу правой руки.
\end{proof}

\begin{remark}
  Если модифицированный алгоритм Грама–Шмидта необходимо применить для
  решения задачи $A\mathbf{x} = \mathbf{b}$, то
  задачу сводят к системе $R\mathbf{x}=\mathbf{d}$.
  При этом матрица $R$ находится указанным способом, а компоненты вектора
  $\mathbf{d}$ для сохранения вычислительной устойчивости определяются по
  следующему алгоритму: $\mathbf{r} = \mathbf{b},\ d_i=(\mathbf{r}_{i-1}, \mathbf{q}_i),\ \mathbf{r}_{i}=\mathbf{r}_{i-1}-d_i\mathbf{q}_i,\ i=1,\ldots,n$.     i   i
\end{remark}

\begin{remark}
  Отметим, что если приближенное решение $\mathbf{x}$ системы $A\mathbf{x}=\mathbf{b}$
  получено каким-либо методом, основанном на $QR$-разложении, то можно выполнить
  следующий процесс уточнения. Найдем вектор невязки $\mathbf{r}=\mathbf{b}-A\tilde{\mathbf{x}}$
  с удвоенным количеством значащих цифр и решим систему $A\mathbf{z}=\mathbf{r}/\norm{\mathbf{r}}$.
  Положим $\tilde{\mathbf{x}}=\tilde{\mathbf{x}}+\norm{\mathbf{r}}\mathbf{z}$.
  Процесс уточнения значительно экономичнее, чем решение
  исходного уравнения, так как разложение матрицы $A$ уже имеется. Уточнение
  можно повторять до тех пор, пока убывает норма вектора невязки.
\end{remark}

Подобное разложение $A=QR$ можно построить и с квадратной матрицей $Q\in\R^{m\times m}$.
\begin{theorem}
  Пусть $m>n$, $\text{rk}(A)=n$. Тогда матрицу $A\in\R^{m\times n}$
  можно привести к виду
  \[A=QR=\left(\begin{array}{cc|c}
        Q_1       &  & Q_2           \\
        m\times n &  & m\times (m-n)
      \end{array}\right)
    \left(\begin{array}{c}
        R_1       \\
        n\times n \\
        \hline
        0         \\
        (m - n) \times n
      \end{array}\right)=Q_1R_1\]
  \[Q\in\R^{m\times m},\ Q^TQ=I,\ R\in\R^{m\times n},\ \det R_1\neq0\]
  $R_1\in\R^{n\times n}$ - верхнетреугольная. В этом
  случае решение $\mathbf{x}$ задачи $R_1\mathbf{x}=Q_1^T\mathbf{b}$ является решением ЗНК.
\end{theorem}
\begin{proof}
  Матрицу $Q_1\in\R^{m\times n}$ мы строим аналогично предыдущей теореме.
  Можем для этого использовать или алгоритм Грама-Шмидта,
  или метод отражений, или метод вращений.
  Для выполнения требования $Q^TQ=I$ в $Q_2$ мы поставим
  дополнительный базис к $Q_1$ из пространства $\R^m$,
  то есть $(m-n)$ векторов, ортонормальные векторам из $Q_1$.

  Покажем, что решение $R_1\mathbf{x}=Q_1^T\mathbf{b}$ является решением ЗНК.
  Действительно, так как решение нормального уравнения является решением ЗНК, то
  \[A^TA\mathbf{x}=A^T\mathbf{b}\Leftrightarrow (QR)^T(QR)\mathbf{x}=(QR)^T\mathbf{b}\Leftrightarrow R^TR\mathbf{x}=R^TQ^T\mathbf{b}\Leftrightarrow R\mathbf{x}=Q^T\mathbf{b}\Leftrightarrow R_1\mathbf{x}=Q_1^T\mathbf{b}\]
\end{proof}
