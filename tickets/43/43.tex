\section{Наилучшее приближение в линейном нормированном и гильбертовом пространствах.}
\textbf{Наилучшее приближение в линейном пространстве}
Пусть $L$ - линейное (векторное) пространство с нормой $\Vert\doteq\Vert$.
Пусть $\{g^{(i)}\},\ i=1,\ldots,n$ - лин. нез. набор.
Решаем задачу $f\in L$
\begin{equation}\label{lin::task}
  \inf_{\{c_i\}}\left\Vert f-\sum_{i=1}^nc_ig^{(i)}\right\Vert
\end{equation}
\begin{theorem}
  В линейном полном норм. пр-ве $\exists$ решение \eqref{lin::task}, то есть $\exists\ g^f=\sum_{i=1}^nc_i^fg^{(i)},\ g^f\in<g^{(1)},\ldots,g^{(n)}>$
\end{theorem}
\begin{proof}
  Докажем, что $\inf$ достигается.
  Рассмотрим фиксированный $F_f(c)=\left\Vert f-\sum_{i=1}^nc_ig^{(i)}\right\Vert$ - непрерывен по $\overline{c}=(c_1,\ldots,c_n)^T$, так как
  \begin{equation*}
    \left|F_f(c)-F_f(\tilde{c})\right|=\left|\left\Vert f-\sum_{i=1}^nc_ig^{(i)}\right\Vert-\left\Vert f-\sum_{i=1}^n\tilde{c}_ig^{(i)}\right\Vert\right|\leq\\
    \leq \left\Vert \sum_{i=1}^n(c_ig^{(i)}-\tilde{c}_ig^{(i)})\right\Vert\leq\sum_{i=1}^n\left|c_i-\tilde{c}_i\right|\Vert g^{(i)}\Vert\rightarrow0
  \end{equation*}
  Заметим, что $0<\leq\inf_{c\in\R^n}F_f(c)\leq\Vert f\Vert$
\end{proof}

\textbf{Наилучшее приближение в гильбертовом пространстве}
Отличается от полного линейного нормой $\Vert f\Vert=(f,f)^{\frac{1}{2}}$,
то есть $f\in H$ - бесконечномерное нормированное пространство.
$\{g^{(i)}\}^n_{i=1}$ - л.н. набор. Тогда
$\exists$ решение \eqref{lin::task}, найдем его:
\begin{equation*}
  \hat{f}_f=\left\Vert f-\sum_{i=1}^nc_ig^{(i)}\right\Vert^2 = \left(f-\sum_{i=1}^nc_ig^{(i)}, f-\sum_{i=1}^nc_ig^{(i)}\right)= \\
  = (f,f)-2\left(\sum_{i=1}^nc_ig^{(i)}, f\right)+\left(\sum_{i=1}^nc_ig^{(i)}, \sum_{i=1}^nc_ig^{(i)}\right)\rightarrow\inf_{\{c_i\}}
\end{equation*}
Но по $\{c_i\}$ -- это обычный квадратичный функционал, а чтобы найти его $\min$ продифф. по $\{c_i\}$:
\begin{equation*}
  \frac{\partial \hat{f}_f}{\partial c_i} = 0\Leftrightarrow  \\
  \underset{G_n}{\begin{pmatrix}
      (g^{(1)},g^{(1)}) & \ldots & (g^{(n)},g^{(1)}) \\
      \ldots            & \ldots & \ldots            \\
      (g^{(1)},g^{(n)}) & \ldots & (g^{(n)},g^{(n)})
    \end{pmatrix}}
  \begin{pmatrix}
    c_1    \\
    \ldots \\
    c_n
  \end{pmatrix} =
  \begin{pmatrix}
    (f,g^{(1)}) \\
    \ldots      \\
    (f,g^{(n)})
  \end{pmatrix}
\end{equation*}
\begin{theorem}
  Пусть $\{g^{(i)}\}_{i=1}^n$ л.н., тогда $G_n=G_n^T>0$, $\deg G_n\neq0$,
  то есть $\forall f\ \exists!\ \mathbf{c}=(c_1,\ldots,c_n)^T$ - решение \eqref{lin::task},
  где $\mathbf{c}$ удовлетворяет системе $G_n\mathbf{c}=\mathbf{b}$,
  ${G_n}_{i,j}=(g^{(i)},g^{(j)})$, $\mathbf{b}=(f,g^{(i)})$.
\end{theorem}
