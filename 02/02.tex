\section{Линейные разностные уравнения n-го порядка.
  Теоремы о представлении общего решения однородного уравнения
  и общего решения неоднородного уравнения.
 }

\begin{definition}
  Пусть неизвестная функция $y$, заданные функции $a$, $f$
  -- функции одного целочисленного аргумента $k$.
  \[ a_0(k)y(k)+a_1(k+1)+\ldots+a_{n-1}(k)y(k+n-1)+a_n(k)y(k+n)=f(k) \Leftrightarrow Ly=f \]
  Тогда уравнение при $f(k),\ a_0(k),\ a_n(k) \neq 0$ называется
  неоднородным линейным разностным уравнением $n$-го порядка.
  Порядок -- количество начальных условий, при которых уравнение
  является разрешимым. Уравнение $Ly = 0$ называется однородным.
\end{definition}
\begin{example}
  \[S(n)=\sum_{i=0}^{n}(1+i+i^3)\underset{\text{переход к разностному уравнению}}{\Rightarrow}S_{n+1}=S_n+1+(n+1)+(n+1)^3\]
  Получили уравнение первого порядка, а в качестве начальных условий возьмем $S_0=+1+(0+1)+(0+1)^3=3$.
\end{example}
\begin{theorem}
  Пусть $y^{(1)}(k),\ldots,y^{(n)}(k)$ - произвольные линейно независимые
  решения линейного однородного разностного уравнения $n$-го порядка
  $Ly=0$, тогда общее решение можно представить в виде
  \[y(k)=\sum_{i=1}^{n}c_iy^{(i)}(k)\]
  $c_i$ в данной записи порождаются из начальных условий задачи.
\end{theorem}
\begin{proof}
  \begin{enumerate}
    \item Покажем, что если функция совпадает на $n$ точках,
          то она совпадает и на всех остальных.

          Так как наше уравнение имеет порядок $n$, то
          $\exists$ ${a_0(k)},\ldots,a_{n-1}(k)\neq0$.

          Перепишем исходное уравнение в двух видах
          \begin{equation}\label{eq:lin:forward}
            y(k+n)=-\sum_{i=0}^{n-1}\frac{a_i(k)}{a_n(k)}y(k+i)
          \end{equation}
          \begin{equation}\label{eq:lin:backward}
            y(k)=-\sum_{i=1}^{n}\frac{a_i(k)}{a_0(k)}y(k+i)
          \end{equation}

          Таким образом, если мы знаем $n$ точек $y(k_0),\ldots,y(k_0+n-1)$,
          то из равенства~\eqref{eq:lin:forward} мы можем восстановить $y(k)\ \forall\ k\geq k_0+n$,
          а из равенства~\eqref{eq:lin:backward} мы можем восстановить $y(k)\ \forall\ k\leq k_0$.
          То есть если мы имеем $n$ точек, то мы можем однозначно восстановить
          все решение, а это значит, что если два решения совпадают на
          $n$ точках, то они тождественно равны $\forall\ k$.
    \item Дано $\left\{y^{(i)}(k)\right\}_{i=1}^{n}$ -- $n$
          линейно независимых решений нашего уравнения. Зафиксировав
          $k=k_0,\ldots,k_{n-1}$ мы получим базис в пространстве $\R^n$.
          В этом базисе мы можем выразить искомое решение как линейную
          комбинацию элементов базиса
          \[y(k)=\sum_{i=1}^{n}c_iy^{(i)}(k),\ k=k_0,\ldots,k_{n-1}\]
          Из предыдущего пункта знаем, что если решение совпадает на $n$
          точках, то совпадает и везде, то есть
          \[y(k)=\sum_{i=1}^{n}c_iy^{(i)}(k),\ \forall k\]
  \end{enumerate}
\end{proof}

\begin{definition}
  \textit{Общим} решением задачи называют то решение, которое
  можно получить из любых начальных условий.
\end{definition}

\begin{theorem}
  Пусть $y^o(k)$ - общее решение однородной задачи $Ly=0$.
  Пусть $y^1(k)$ - некоторое частное решение неоднородной
  задачи $Ly=f$. Тогда любое решение неоднородной задачи
  можно представить в виде
  \[y(k)=y^o(k)+y^1(k)\]
\end{theorem}
\begin{proof}
  Пусть $y(k)$ -- какое-либо решение задачи $Ly=f$,
  $y^1(k)$ -- некоторое частное решение этой же задачи. Тогда
  $y(k)-y^1(k)$ является решением задачи $Ly=0$:
  \[L(y(k)-y^1(k))=L(y(k))-L(y^1(k))=f(k)-f(k)=0\Rightarrow y(k)=y^o(k)+y^1(k)\]
\end{proof}
